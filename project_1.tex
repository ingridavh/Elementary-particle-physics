\documentclass[11pt]{article}
\usepackage{graphicx}
\usepackage[utf8]{inputenc} 
\usepackage{amsmath}
\usepackage{cancel}
\usepackage{bbold}
\usepackage{color}
\usepackage{amsfonts}
\usepackage{mathtools}
\usepackage{braket}
\usepackage{float}
\usepackage{lscape}
\usepackage{multicol}
\usepackage{tikz-feynman}
\usepackage{tikz}
\usepackage{subcaption}
\usepackage{multicol}

\usepackage{geometry}
\geometry{legalpaper, margin=0.5in}

\begin{document}
\title{FYS4560 Project 1}
\author{Ingrid A V Holm}
\maketitle

\begin{multicols}{2}
\section{Standard model and beyond}
\subsection*{- Allowed, forbidden and discovery processes}

\begin{flushleft}
We consider some processes to state whether they are allowed or not. Quantities that should be conserved include
\begin{itemize}
\item Lepton number
\item Baryon number
\item Energy and momentum
\item Isospin
\item Parity
\item Charge parity
\item Strong interactions are only between quarks and gluons.
\item The Higgs boson interacts through the weak interaction, $Z, W^{\pm}$, or fermionic coupling?
\end{itemize}
\end{flushleft}

\subsection*{Electron-positron collisions}

\subsection{$e^+e^- \rightarrow q \bar{q} gg$}
\begin{flushleft}
Can interact through a combination of electroweak and strong interaction, but requires 4 vertices, so the crossection is small
\begin{figure}[H]
\centering
\begin{tikzpicture}
\begin{feynman}[small] 
\vertex (a) {\(e^-\)};
\vertex [below right=of a] (b) ; 
\vertex [below left=of b] (c) {\(e^+\)}; 
\vertex [right=of b] (d); 
\vertex [above right=of d] (e); 
\vertex [above right=of e] (g) {\(q\)}; 
\vertex [below=of g] (p) {\(g\)}; 
\vertex [below right=of d] (f); 
\vertex [below right=of f] (h) {\(\bar{q}\)};
\vertex [above=of h] (p2) {\(g\)};
\diagram{ (a) --[fermion] (b) --[fermion] (c); 
(b) --[photon, edge label={\(\gamma\)}] (d) --[fermion] (e) --[fermion] (g);
(e) --[gluon] (p); (d) --[fermion] (f) --[fermion] (h);
(f) --[gluon] (p2)
};
\end{feynman}
\end{tikzpicture}
\begin{tikzpicture}
\begin{feynman}[small] 
\vertex (a) {\(e^-\)};
\vertex [below right=of a] (b) ; 
\vertex [below left=of b] (c) {\(e^+\)}; 
\vertex [right=of b] (d); 
\vertex [above right=of d] (e); 
\vertex [above right=of e] (g) {\(q\)}; 
\vertex [below right=of e] (p) ; 
\vertex [below right=of p] (p1);
\vertex [above right= of p] (p2);
\vertex [below right=of d] (f); 
\vertex [below right=of f] (h) {\(\bar{q}\)};
\diagram{ (a) --[fermion] (b) --[fermion] (c); 
(b) --[photon, edge label={\(\gamma\)}] (d) --[fermion] (e) --[fermion] (g);
(e) --[gluon, edge label={\(g\)}] (p); (d) --[fermion] (h); (p) --[gluon, edge label={\(g\)}] (p1); (p) --[gluon,edge label={\(g\)}] (p2)
};
\end{feynman}
\end{tikzpicture}
\caption{Possible diagrams for $e^+e^- \rightarrow q \bar{q} gg$. The diagram on the left can also have both gluons on one 'arm'.}
\end{figure}
\end{flushleft}
\subsection{$e^+e^- \rightarrow \tilde{l}^+ \tilde{l}^-$}
Is this a supersymmetric lepton? This decay can happen through the electromagnetic interaction QED. \textit{Allowed.}

\begin{figure}[H]
\centering
\begin{tikzpicture}
\begin{feynman}[small]
\vertex (a) {\(e^-\)}; \vertex [below right=of a] (b); \vertex [below left=of b] (c) {\(e^+\)};
\vertex [right=of b] (d); \vertex [above right=of d] (e) {\( \tilde{l}^+\)}; \vertex [below right=of d] (f) {\( \tilde{l}^-\)};
\diagram{ (a) -- [fermion] (b) --[fermion] (c); (b) --[photon, edge label= {\(\gamma, Z\)}] (d); (f) --[scalar] (d) --[scalar] (e);
};
\end{feynman}
\end{tikzpicture}
\caption{Diagram for $e^+e^- \rightarrow \tilde{l}^- \tilde{l}^+$.}
\end{figure}

\subsection{$e^+ e^- \rightarrow HH \gamma$}
\begin{flushleft}
Allowed through the weak and electroweak interaction. There are three $5$-vertex diagrams, but two of them include a $WW \gamma$-vertex, which has a second order coupling constant.
\begin{figure}[H]
\centering
\begin{tikzpicture}
\begin{feynman}[small]
\vertex (r1) {\(e^-\)}; \vertex [below right=of r1] (r2); \vertex [below left=of r2] (r3) {\(e^+\)}; \vertex [right=of r2] (p); \vertex [above right=of p] (p1); \vertex [below right= of p] (p2); \vertex [above right=of p2] (k); \vertex [right=of k] (m);
\vertex [above=of m] (t) {\(H^0\)}; \vertex [below=of m] (b) {\(H^0\)};
\diagram{ (r1) --[fermion] (r2) --[fermion](r3); (r2) --[photon, edge label={\(Z, \gamma\)}] (p); (p) --[fermion, edge label={\(t\)}] (p1) --[fermion, edge label={\(t\)}] (k) --[fermion, edge label={\(t\)}] (p2) --[fermion, edge label={\(t\)}] (p); (p1) --[scalar] (t); (p2) --[scalar] (b); (k) --[photon, edge label={\(\gamma\)}] (m)
};
\end{feynman}
\end{tikzpicture}
\caption{Diagram for $e^+ e^- \rightarrow HH \gamma$.}
\end{figure}
\end{flushleft}

\subsection{$e^+ e^- \rightarrow ZZZ$}
\begin{flushleft}
Important because it's a precision test of vertices containing to $Z$ bosons. Since this process creates two jets; one where two $Z$'s come out, and one where one comes out, we see that $ZZZ$-vertices are not observed. It could also occur via the electroweak interaction, but then the resulting bosons would be $\gamma ZZ$ or $\gamma \gamma \gamma$.
\begin{figure}[H]
\centering
\begin{tikzpicture}
\begin{feynman}[small]
\vertex (r1) {\(e^-\)}; \vertex [below right=of r1] (r2); \vertex [below left=of r2] (r3) {\(e^+\)}; \vertex [right= of r2] (p); \vertex [below right=of p] (p1); \vertex [above right = of p1] (p11) {\(Z\)}; \vertex [below right= of p1] (p12) {\(Z\)}; \vertex [above=of p11] (p2) {\(Z\)};
\diagram{(r1) --[fermion] (r2) --[fermion] (r3); (r2) --[photon, edge label={\(Z\)}] (p); (p) --[scalar, half left, edge label={\(W\)}] (p1) --[scalar, half left, edge label={\(W\)}] (p); (p11) --[photon] (p1) --[photon] (p12); (p) --[photon] (p2)
};
\end{feynman}
\end{tikzpicture}
\caption{Diagram for $e^+ e^- \rightarrow ZZZ$.}
\end{figure}
\end{flushleft}
\subsection{$e^+e^- \rightarrow H \rightarrow gg$}
This process is allowed via the strong and the weak interaction, and produces a Higgs boson which then decays into top quark production.
\begin{figure}[H]
\centering
\begin{tikzpicture}
\begin{feynman}[small]
\vertex (r1) {\(e^-\)}; \vertex[right =of r1] (p1); \vertex[below right=of p1] (k); \vertex [below left=of k] (p2); \vertex [left=of p2] (r2) {\(e^+\)};  \vertex [right=of k] (m); \vertex[above right = of m] (m1); \vertex [below right=of m] (m2); \vertex [right = of m1] (m11) {\(g\)}; \vertex [right=of m2] (m22) {\(g\)};
\diagram{
(r1) --[fermion] (p1) --[fermion] (p2) --[fermion] (r2); 
(p1) --[photon, edge label={\(Z, \gamma\)}] (k) --[photon, edge label={\(Z, \gamma\)}] (p2); 
(k) --[scalar, edge label={\(H^0\)}] (m);
 (m) --[fermion, edge label={\(t\)}] (m1) --[fermion, edge label={\(t\)}] (m2) --[fermion, edge label={\(t\)}] (m);
(m1) --[gluon] (m11); (m2) --[gluon] (m22) 
};
\end{feynman}
\end{tikzpicture}
\caption{Diagram for $e^-e^+ \rightarrow H \rightarrow gg$.}
\end{figure}
\begin{flushleft}

\end{flushleft}

\subsection{$e^+e^- \rightarrow \nu \bar{\nu} \gamma \gamma$}
\begin{flushleft}
This process is allowed through the electroweak interaction.
\begin{figure}[H]
\begin{tikzpicture}
\begin{feynman}[small]
\vertex (r1) {\(e^-\)}; \vertex [below=of r1] (r2) {\(e^+\)}; \vertex [right = of r1] (p1); \vertex [right=of r2] (p2);
\vertex [right =of p2] (k2); \vertex[right =of p1] (k1);
\vertex [right=of k1] (m1) {\(\nu\)}; \vertex [above =of m1] (m11) {\(\gamma\)}; \vertex [right= of k2] (m2) {\(\bar{\nu}\)}; \vertex [below=of m2] (m22) {\(\gamma\)};
\diagram{ (r1) --[fermion] (p1) --[fermion] (k1) --[fermion] (m1); (m2) --[fermion] (k2) --[fermion] (p2) --[fermion] (r2);
(p1) --[photon] (m11); (p2) --[photon] (m22); (k1) --[scalar, edge label={\(W\)}] (k2)
};
\end{feynman}
\end{tikzpicture}
\begin{tikzpicture}
\begin{feynman}[small]
\vertex (r1) {\(e^-\)}; 
\vertex [right =of r1] (p1); 
\vertex [below right=of p1] (k); 
\vertex [below left= of k] (p2); 
\vertex [left=of p2] (r2) {\(e^+\)};
\vertex [above right=of k] (k1) {\(\gamma\)}; 
\vertex [below right= of k] (k2) {\(\gamma\)}; 
\vertex [above= of k1] (k11) {\(\nu\)}; 
\vertex [below=of k2] (k22) {\(\bar{\nu}\)};
\diagram{(r1) --[fermion] (p1) --[fermion] (k11); 
(k22) --[fermion] (p2) --[fermion] (r2);
(p1) --[scalar, edge label={\(W\)}] (k) --[photon] (k1); (k2)--[photon] (k)--[scalar, edge label={\(W\)}] (p2)
};
\end{feynman}
\end{tikzpicture}
\caption{Diagrams for $e^+e^- \rightarrow \nu \bar{\nu} \gamma \gamma$.}
\end{figure}
\end{flushleft}

\subsection{$e^+e^- \rightarrow Y(3s) \rightarrow B^0 \bar{B}^0$}
\begin{flushleft}
This process is allowed through the electroweak interaction, or a combination of the electroweak and strong interaction.
\begin{figure}[H]
\centering
\begin{tikzpicture}
\begin{feynman}[small]
\vertex (r1); \vertex [below right= of r1] (r); \vertex [below left=of r] (r2); \vertex [right=of r] (p); 
\vertex [right of= p] (k); \vertex [right of= k] (k2); \vertex [above right=of k2] (k21) {\(d\)}; \vertex [below=of k21] (k22) {\(\bar{d}\)}; \vertex [above=of k21] (k1) {\(\bar{b}\)}; \vertex [below=of k22] (k3) {\(b\)};
\diagram{ (r1) --[fermion] (r)--[fermion] (r2); (r) --[photon, edge label={\(\gamma, Z\)}] (p); (k1) --[fermion] (k) --[fermion] (p) --[fermion] (k3); (k2) --[photon, edge label={\(\gamma, Z, g\)}] (k); (k22) --[fermion] (k2) --[fermion] (k21)
};
\end{feynman}
\end{tikzpicture}
\caption{Diagram for $e^+e^- \rightarrow Y(3s) \rightarrow B^0 \bar{B}^0$. $Y(3s)$ is made up of $\bar{b}b $.}
\end{figure}
\end{flushleft}

\subsection{$e^+ e^- \rightarrow Z^0 t \bar{t}$}

\subsection*{Gluon-gluon fusion}

\subsection{$gg \rightarrow e^+e^-$}
\begin{flushleft}
This interaction is possible, but requires four vertices so its cross section must be small. It is a combination of the electroweak and strong interaction.
\end{flushleft}
\begin{figure}[H]
\centering
\begin{tikzpicture}
\begin{feynman}[small]
\vertex (a) {\(g\)}; \vertex [below right=of a] (b); \vertex [below left=of b] (c) {\(g\)}; \vertex [right=of b] (d); \vertex [right= of d] (e);
\vertex [right= of e] (f); \vertex [above right=of f] (g) {\(e^-\)}; 
\vertex [below right= of f ] (h) {\(e^+\)};
\diagram{ (a) --[gluon] (b) --[gluon] (c); (b) -- [gluon] (d); (d) --[fermion, half left, edge label={\(q\)}] (e) --[fermion, half left, edge label={\(q\)}] (d);(e) --[photon, edge label={\(\gamma, Z\)}] (f); (h) --[fermion] (f) --[fermion] (g)
};
\end{feynman}
\end{tikzpicture}
\begin{tikzpicture}
\begin{feynman}[small]
\vertex (r); \vertex [above left=of r] (r1); \vertex [left=of r1] (r11) {\(g\)}; \vertex [below left=of r] (r2); \vertex [left=of r2] (r22) {\(g\)};
\vertex [right = of r] (p); \vertex [above right=of p] (p1) {\(e^-\)}; \vertex [below right=of p] (p2) {\(e^+\)};
\diagram{ (r) --[fermion, edge label={\(q\)}] (r2) --[fermion, edge label={\(q\)}] (r1) --[fermion, edge label={\(q\)}] (r); (r1) --[gluon] (r11); (r2) --[gluon] (r22); (r) --[photon, edge label={\(\gamma, Z\)}] (p); (p2) --[fermion] (p) --[fermion] (p1)
};
\end{feynman}
\end{tikzpicture}
\caption{Diagrams for $gg \rightarrow e^- e^+$.}
\end{figure}

\subsection{$gg \rightarrow t \bar{t} HH$}
\begin{flushleft}
This process is allowed via the strong interaction. It's important in order to determine whether there are $HH$-vertices.
\begin{figure}[H]
\centering
\begin{tikzpicture}
\begin{feynman}[small]
\vertex (r1) {\(g\)}; \vertex [right=of r1] (p1) ; \vertex [below=of p1] (p2) ; \vertex [left=of p2] (r2) {\(g\)};
\vertex [right=of p1] (k1); \vertex [above right=of k1] (k11) {\(\bar{t}\)};\vertex [below= of k11] (k12) {\(H^0\)}; \vertex [right= of p2] (k2); \vertex [below right= of k2] (k22) {\(t\)}; \vertex [above= of k22] (k21) {\(H^0\)};
\diagram{ (r1) --[gluon] (p1); (r2) --[gluon] (p2); (k11) --[fermion] (k1) --[fermion, edge label={\(t\)}] (p1) --[fermion, edge label={\(t\)}] (p2) -- [fermion, edge label={\(t\)}] (k2) --[fermion] (k22); (k2) --[scalar] (k21); (k1) --[scalar] (k12)
};
\end{feynman}
\end{tikzpicture}
\caption{Diagram for $gg \rightarrow t \bar{t} HH$.}
\end{figure}
\end{flushleft}

\subsection{$gg \rightarrow H \rightarrow Z \gamma$}
\begin{flushleft}
This process is allowed through a combination of the strong and electroweak interaction, using a top quark loop. This diagram is important for the top quark production and decay.
\begin{figure}[H]
\begin{tikzpicture}
\begin{feynman}
\vertex (r1) {\(g\)}; \vertex [right=of r1] (p1) ; \vertex [below=of p1] (p2) ; \vertex [left=of p2] (r2) {\(g\)};
\vertex [below right= of p1] (p); \vertex [right=of p] (k); \vertex [above right=of k] (k1); \vertex [below=of k1] (k2); \vertex [right =of k1] (k11) {\(Z, \gamma\)}; \vertex [right=of k2] (k22) {\(\gamma, Z\)};
\diagram{ (r1) --[gluon] (p1) --[fermion, edge label={\(t\)}] (p) --[fermion, edge label={\(t\)}] (p2) --[fermion, edge label={\(t\)}] (p1); (r2) --[gluon] (p2); (p) --[scalar, edge label={\(H^0\)}] (k); (k) --[scalar, edge label={\(W\)}] (k1) --[scalar, edge label={\(W\)}] (k2) --[scalar, edge label={\(W\)}] (k); (k1) --[photon] (k11); (k2) --[photon] (k22)
};
\end{feynman}
\end{tikzpicture}
\caption{Diagrams for $gg \rightarrow H \rightarrow Z \gamma$.}
\end{figure}
\end{flushleft}


\subsection*{Quark-antiquark collisions}

\subsection{$q \bar{q} \rightarrow W^+ W^- Z$}
\begin{flushleft}
This process is allowed through the weak interaction. It is important because it gives evidence for the existence of the $WWZZ$-four vertex.
\begin{figure}[H]
\centering
\begin{tikzpicture}
\begin{feynman}[small]
\vertex (r1) {\(q\)}; \vertex [below right=of r1] (r2); \vertex [below left=of r2] (r3) {\(\bar{q}\)}; \vertex[right=of r2] (p); \vertex [right =of p] (k2) {\(Z\)}; \vertex [above=of k2] (k1) {\(W\)}; \vertex [below=of k2] (k3) {\(W\)};
\diagram{(r1) --[fermion] (r2) --[fermion] (r3); (r2) --[photon, edge label={\(Z\)}] (p); (k1) --[scalar] (p) --[scalar] (k3); (p) --[photon] (k2)
};
\end{feynman}
\end{tikzpicture}
\end{figure}
\end{flushleft}


\subsection{$q \bar{q} \rightarrow gg e^+ e^-$}
\begin{flushleft}
This process is allowed through a combination of the strong and electroweak interactions.
\begin{figure}[H]
\centering
\begin{tikzpicture}
\begin{feynman}[small]
\vertex (r1) {\(q\)}; \vertex [right= of r1] (p1); \vertex[below right= of p1](p); \vertex [below left= of p] (p2); \vertex [left = of p2] (r2) {\(\bar{q}\)}; \vertex [right= of p] (k); \vertex [above right=of k] (k1) {\(e^-\)}; \vertex [below right= of k] (k2) {\(e^+\)}; \vertex[above= of k1] (k0) {\(g\)} ;\vertex [below= of k2] (k3) {\(g\)};
\diagram{ (r1) --[fermion] (p1) --[fermion] (p) --[fermion] (p2) --[fermion] (r2); (p1)--[gluon] (k0); (p2) --[gluon] (k3); (k2) --[fermion] (k) --[fermion] (k1); (p)--[photon, edge label={\(\gamma, Z\)}] (k)
};
\end{feynman}
\end{tikzpicture}
\begin{tikzpicture}
\begin{feynman}[small]
\vertex (r1) {\(q\)}; \vertex [right=of r1] (p1); \vertex [below=of p1] (p2); \vertex [left=of p2] (r2) {\(\bar{q}\)}; \vertex [right=of p1] (k1); \vertex[above right=of k1] (k11) {\(g\)}; \vertex [below =of k11] (k12) {\(g\)}; \vertex [right =of p2] (k2); \vertex [below right=of k2] (k22) {\(e^+\)}; \vertex [above=of k22] (k21) {\(e^-\)};
\diagram{ (r1) --[fermion] (p1) --[fermion] (p2) --[fermion] (r2); (p1) --[gluon] (k1) --[gluon] (k11); (k1) --[gluon] (k12); (p2) --[photon, edge label={\(\gamma, Z\)}] (k2); (k22) --[fermion] (k2)--[fermion](k21)
};
\end{feynman}
\end{tikzpicture}
\end{figure}
\end{flushleft}


\subsection*{Proton-proton and proton-antiproton collisions}

\subsection{$p \bar{p} \rightarrow l^+ l^- X$}
\begin{flushleft}
Quark and antiquark from proton and antiproton can annihilate and become lepton and antilepton through a virtual photon. \textit{Allowed.}
\begin{figure}[H]
\centering
\begin{tikzpicture}
\begin{feynman}[small]
\vertex (r); \vertex [above left=of r] (r1) {\(q\)}; \vertex [below left=of r] (r2) {\(\bar{q}\)}; \vertex [right= of r] (p); \vertex [above right=of p] (p1) {\(l^-\)}; \vertex [below right= of p] (p2) {\(l^+\)};
\diagram{ (r1) --[fermion] (r) --[fermion] (r2); (r) --[photon, edge label={\(\gamma, Z\)}] (p); (p2) --[fermion] (p) --[fermion] (p1)
};
\end{feynman}
\end{tikzpicture}
\caption{Diagram for $p \bar{p} \rightarrow l^+ l^-$.}
\end{figure}
\end{flushleft}

\subsection{$pp \rightarrow l^+ l^- l^+ l^- X$}
\begin{flushleft}
This process is important because it requires the production of a Higgs boson. It is allowed via the weak interaction.
\begin{figure}[H]
\centering
\begin{tikzpicture}
\begin{feynman}[small]
\vertex (r1) {\(q\)}; \vertex [right=of r1] (p1);
\vertex [below right= of p1] (p2); 
\vertex [below left=of p2] (p3); 
\vertex [left=of p3] (r3) {\(q\)}; 
\vertex [right= of p2] (k);
\vertex [above right = of k] (k1);
\vertex [above right = of k1] (k11) {\(l^-\)};
\vertex [below= of k11] (k12) {\(l^+\)};
\vertex [below right = of k] (k2);
\vertex [above right = of k2] (k21) {\(l^-\)};
\vertex [below= of k21] (k22) {\(l^+\)};
\vertex [above = of k11] (t) {\(q\)};
\vertex [below = of k22] (b) {\(q\)};
\diagram{ (r1) --[fermion] (p1) --[fermion] (t); (r3) --[fermion] (p3) --[fermion] (b); (p1) -- [photon, edge label={\(Z\)}](p2) --[photon, edge label={\(Z\)}] (p3); (p2) --[scalar, edge label={\(H^0\)}] (k); (k2) --[photon, edge label={\(Z\)}] (k) --[photon, edge label={\(Z\)}] (k1); (k12) --[fermion] (k1) --[fermion] (k11); (k22) --[fermion] (k2) --[fermion] (k21)
};
\end{feynman}
\end{tikzpicture}
\caption{Diagram for $p \bar{p} \rightarrow l^+ l^-$.}
\end{figure}
\end{flushleft}

\subsection*{Collisions}

\subsection{$K^- p \rightarrow \Omega^- K^+ K^0$}
\begin{flushleft}
This interaction is allowed through the strong interaction.
\begin{figure}[H]
\centering
\begin{tikzpicture}
\begin{feynman}[small]
\vertex (r); 
\vertex [above left= of r] (r1) {\(u\)}; 
\vertex [below left= of r] (r2) {\(\bar{u}\)}; 
\vertex [right=of r] (p); 
\vertex [above right=of p] (p1) {\(s\)}; 
\vertex [below right=of p] (p2) {\(\bar{s}\)};
\vertex [below=of r2] (m) {\(d\)}; 
\vertex [right=of m] (n); 
\vertex [below= of p2] (n2) {\(d\)};
\vertex [below=of n] (o);  
\vertex [below =of n2] (o2) {\(\bar{s}\)};
\vertex [below =of o2] (o1) {\(s\)};
\diagram{ (r2) --[fermion] (r) --[fermion] (r1); (r) --[gluon, edge label={\(g, \gamma\)}] (p); (p2) --[fermion] (p) --[fermion] (p1); (m) --[fermion] (n) --[fermion] (n2); (n) --[gluon, edge label={\(g, \gamma\)}] (o); (o2) --[fermion] (o) --[fermion] (o1)
};
\end{feynman}
\end{tikzpicture}
\begin{tikzpicture}
\begin{feynman}[small]
\vertex [below=of r2](R); 
\vertex [above left=of R] (R1) {\(\bar{u}\)}; 
\vertex [below left=of R] (R2) {\(u\)}; 
\vertex [right=of R] (P); 
\vertex [above right=of P] (P1); 
\vertex [below right=of P] (P2); 
\vertex [above right = of P1] (P11) {\(\bar{s}\)}; 
\vertex [below=of P11] (P12) {\(s\)}; 
\vertex [below right=of P2] (P22) {\(s\)}; 
\vertex [above=of P22] (P21) {\(\bar{s}\)};
\diagram{(R2) --[fermion] (R) --[fermion] (R1); 
(R) --[gluon, edge label={\(g\)}] (P); 
(P1) --[gluon, edge label={\(g\)}] (P) --[gluon, edge label={\(g\)}] (P2); 
(P11) --[fermion] (P1) --[fermion] (P12);
(P21) --[fermion] (P2) --[fermion] (P22)};
\end{feynman}
\end{tikzpicture}
\caption{Diagrams for $K^- p \rightarrow \Omega^- K^+ K^0$.}
\end{figure}
\end{flushleft}

\subsection{$\nu_e e^- \rightarrow \nu_e e^- \gamma$}
\begin{flushleft}
Allowed through electroweak interaction.
\begin{figure}[H]
\begin{tikzpicture}
\begin{feynman}
\vertex (r1) {\(e^-\)}; 
\vertex [right= of r] (p1); 
\vertex [below =of p1] (p2); 
\vertex [below= of p2] (p3); 
\vertex [left=of p3] (r3) {\(\nu_e\)}; 
\vertex [right =of p2] (l); 
\vertex [right= of l] (k2); 
\vertex [above=of k2] (k1) {\(e^-\)}; 
\vertex [below =of k2] (k3) {\(\nu_e\)};
\diagram{ (r1) --[fermion] (p1) --[fermion] (k1); (p3) -- [photon, edge label={\(Z,\gamma\)}] (p2) --[photon, edge label={\(Z, \gamma\)}] (p1); (r3) --[fermion] (p3) --[fermion] (k3); (p2) --[scalar, half left, edge label={\(W\)}] (l) --[scalar, half left, edge label={\(W\)}] (p2); (l) --[photon, edge label={\(\gamma\)}] (k2)
};
\end{feynman}
\end{tikzpicture}
\caption{Diagram for $\nu_e e^- \rightarrow \nu_e e^- \gamma$.}
\end{figure}
\end{flushleft}

\subsection{$ep \rightarrow J/ \psi +X$}
\begin{flushleft}
The $J/ \psi$ meson consists of $c \bar{c}$ quarks, a proton consists of $uud$ and a neutron consists of $udd$. So in order for the process to occur , an up quark must turn into a down quark. This process is allowed through the weak interaction. It's important because it shows that flavour is not conserved in weak interactions. 
\begin{figure}[H]
\begin{tikzpicture}
\begin{feynman}
\vertex (r1) {\(u\)}; \vertex [right =of r1] (p1); \vertex[below right= of p1] (p); \vertex [below left= of p] (p2); \vertex [left=of p2] (r2) {\(e^-\)}; \vertex [right=of p] (k); \vertex[above right = of k] (k1) {\(c\)}; \vertex [above=of k1] (k0) {\(d\)}; \vertex[below right=of k] (k2) {\(\bar{c}\)}; \vertex [below= of k2] (k3) {\(\nu_e\)};
\diagram{ (r1) --[fermion] (p1) --[scalar, edge label={\(W\)}] (p) --[scalar, edge label={\(W\)}] (p2) --[fermion] (r2); (p1) --[fermion] (k0); (p2) --[fermion] (k3); (p) --[photon, edge label={\(Z\)}] (k); (k2) --[fermion] (k) --[fermion] (k1)
};
\end{feynman}
\end{tikzpicture}
\end{figure}
\end{flushleft}


\subsection*{Decays}

\subsection{$\tau^+ \rightarrow \mu^+ \nu_e \bar{\nu}_{\tau}$}
\begin{flushleft}
This process is not allowed because it violates lepton flavour conservation. If we changed $\nu_e$ by $\nu_{\mu}$ it would be allowed.
\end{flushleft}

\subsection{$D^0 \leftrightarrow \bar{D}^0$}
\textit{Allowed.} Weak current or something.

\pagebreak


%-----------START DEL 2


\section{Top quark and $W$-boson}

\subsection*{CKM-matrix}
\begin{flushleft}
The electroweak interaction does not preserve quark flavour. Quarks can change flavour by emitting a $W^{\pm}$-boson. The flavour-changes occur according to the classification of quarks into upper and lower, and the doublets
\begin{align*}
\begin{pmatrix}
u\\
d'\\
\end{pmatrix},
\begin{pmatrix}
c\\
s'\\
\end{pmatrix},
\begin{pmatrix}
t\\
b'\\
\end{pmatrix},
\end{align*}
where $d'$,$s'$ and $b'$ are the weak eigenstates constructed from the quarks $d,s,b$
\begin{align*}
\begin{pmatrix}
d'\\
s'\\
b'\\
\end{pmatrix} = 
V_{CKM} \begin{pmatrix}
d\\
s\\
b\\
\end{pmatrix}.
\end{align*}
The matrix $V_{CKM}$ relates the weak interaction eigenstates $d'$ to the energy eigenstates $d$, and is called the \textit{Cabbibo-Kobayashi-Maskawa matrix}. The current best approximation is given by $V_{CKM}$
\begin{align*}
\begin{pmatrix}
V_{ud} & V_{us} & V_{ub}\\
V_{cd} & V_{cs} & V_{cb}\\
V_{td} & V_{ts} & V_{tb}\\
\end{pmatrix}
\simeq \begin{pmatrix}
0.97427 & 0.22534 & 0.00351 \\
0.22520 & 0.97344 & 0.0412 \\
0.00867 & 0.0400 & 0.999146\\
\end{pmatrix}
\end{align*}
\begin{figure}[H]
\begin{tikzpicture}
\begin{feynman}
\vertex (a) {\(q_u\)}; \vertex [right=of a] (b); \vertex [above right=of b] (c); \vertex [below right=of b] (d) {\(q_d\)};
\diagram{(a)--[fermion] (b) --[photon, edge label={\(W^{\pm}\)}] (c); (b) --[fermion] (d)
};
\end{feynman}
\end{tikzpicture}
\centering
\label{fig:: flavour change}
\caption{Flavour-changing weak interaction with quarks.}
\end{figure}
The interactions of quarks and leptons with the $W^{\pm}$ bosons are called charged currents. Parity and charge conjugation are broken, but $CP$ is usually conserved. The top quark mass $m_t = 173$ GeV, however, is so large that it's production with the bottom quark
\begin{align*}
W^- \rightarrow t b',
\end{align*}
is kinematically forbidden. We will consider some ways the top quark is produced.
\end{flushleft}

\subsection*{$B^0-\bar{B}^0$-oscillations}
\begin{flushleft}
A manifestation of the neutral particle oscillations and charged current, is the $B^0-\bar{B}^0$ oscillations between particle and antiparticle. The neutral $B^0$ meson consists of a strange antiquark $s$ and a bottom quark $b$, or their antiparticles. The mixing has been observed at Fermilab in 2006 and by LHCb at CERN in 2011. The oscillation is important because it can tell us something about the excess of matter (as opposed to antimatter) in the universe. The $V_{CKM}$-matrix elements that go into the oscillations are $V_{tb}$ and $V_{ts}$ for the $B_s$ meson, and $V_{tb}$ and $V_{td}$ for the $B$ meson, as can be seen in Fig. (\ref{fig:: b-bbar mixing}).
\begin{figure}[H]
\centering
\begin{tikzpicture}
\begin{feynman}
\vertex (r1) {\(\bar{b}\)}; 
\vertex [below=of r1] (r2) {\(d\)}; 
\vertex [right=of r1] (p1); 
\vertex [below=of p1] (p2); 
\vertex [right=of p1] (k1);
\vertex [below=of k1] (k2);
\vertex [right=of k1] (q1) {\(\bar{d}\)};
\vertex [below=of q1] (q2) {\(b\)};
\diagram{ (r2) --[fermion] (p2) --[fermion, edge label={\((u,c),t\)}] (p1) --[fermion] (r1); (p1) --[scalar, edge label={\(W\)}] (k1); (p2) --[scalar, edge label={\(W\)}] (k2); (q1) --[fermion] (k1) --[fermion, edge label={\((u,c),t\)}] (k2) --[fermion] (q2)
};
\draw [decoration ={brace}, decorate] (r2.south west)-- (r1.north west) node [pos=0.5, left] {\(B^{0}\)};
\draw [decoration ={brace}, decorate] (q1.north east)-- (q2.south east) node [pos=0.5, right] {\(\bar{B}^{0}\)};
\end{feynman}
\end{tikzpicture}
\begin{tikzpicture}
\begin{feynman}
\vertex (r1) {\(\bar{b}\)}; 
\vertex [below=of r1] (r2) {\(s\)}; 
\vertex [right=of r1] (p1); 
\vertex [below=of p1] (p2); 
\vertex [right=of p1] (k1);
\vertex [below=of k1] (k2);
\vertex [right=of k1] (q1) {\(\bar{s}\)};
\vertex [below=of q1] (q2) {\(b\)};
\diagram{ (r2) --[fermion] (p2) --[fermion, edge label={\((u,c),t\)}] (p1) --[fermion] (r1); (p1) --[scalar, edge label={\(W\)}] (k1); (p2) --[scalar, edge label={\(W\)}] (k2); (q1) --[fermion] (k1) --[fermion, edge label={\((u,c),t\)}] (k2) --[fermion] (q2)
};
\draw [decoration ={brace}, decorate] (r2.south west)-- (r1.north west) node [pos=0.5, left] {\(B^{0}_s\)};
\draw [decoration ={brace}, decorate] (q1.north east)-- (q2.south east) node [pos=0.5, right] {\(\bar{B}^{0}_s\)};
\end{feynman}
\end{tikzpicture}
\caption{Diagrams for $B^0-\bar{B}^0$ and $B^0_s-\bar{B}^0_s$ oscillations.}
\label{fig:: b-bbar mixing}
\end{figure}
\end{flushleft}


\subsection*{Top quark decay}

\begin{flushleft}
The top quark is so heavy and shortlived that it does not have time to form hadrons before it decays. It decays through the weak interaction to a $W$-boson and a down-type quark, i.e. down, strange or bottom.
\begin{align*}
t \rightarrow W^+ b,s,d
\end{align*}

\begin{figure}[H]
\centering
\begin{tikzpicture}
\begin{feynman}
\vertex (a) {\(t\)}; \vertex [right=of a] (b); \vertex [above right= of b] (c); \vertex [below right=of b] (d) {\(d,s,b\)};
\diagram{
(a) --[fermion] (b) --[photon, edge label= {\(W^+\)}] (c), (b) --[fermion] (d)};
\end{feynman}
\end{tikzpicture}
\caption{Top quark decay}
\end{figure}
\end{flushleft}


\section{Gauge theories}

\subsection*{Standard model}
\begin{flushleft}
The particles in the Standard model are organized in a threefold family structure
\begin{align*}
\begin{pmatrix}
\nu_e & u\\
e^- & d'\\
\end{pmatrix}, 
\begin{pmatrix}
\nu_{\mu} & c\\
\mu^- & s'\\
\end{pmatrix},
\begin{pmatrix}
\nu_{\tau} & t\\
\tau^- & b'\\
\end{pmatrix}.
\end{align*}
\end{flushleft}

\subsection*{Gauge and symmetries}

\begin{flushleft}
The gauge principle is the requirement that a Lagrangian, e.g. the free Dirac fermion Lagrangian
\begin{align*}
\mathcal{L}_0 = \bar{\psi}(i \partial_{\mu} - m) \psi,
\end{align*}
which is symmetric under $U(1)$ \textit{global} transformations also be \textit{symmetric under local transformations}.
\end{flushleft}

\begin{flushleft}
The standard model is based on a non-Abelian group of transformations, called the Lorentz group, or $SU(3) \otimes U(1)$. The spinors are symmetric under local transformations
\begin{align*}
\psi \rightarrow \psi'=e^{iQ\theta} \Psi,
\end{align*}
and the theory is gauged by promoting these symmetries to local symmetries, i.e.
\begin{align*}
\theta \rightarrow \theta (x).
\end{align*}
In order to preserve symmetries we must introduce the covariant derivative
\begin{align}
D_{\mu}(x) = \partial_{\mu} - ieQA_{\mu}(x)
\end{align}
\end{flushleft}

\begin{flushleft}
Symmetries are related to conserved quantities through Noether's theorem, which claims that for every symmetry there must be a conserved quantity.
\end{flushleft}



\subsection{QCD}
\begin{flushleft}
From experiments we know that hadronic matter is made up of quarks -- mesons are $q \bar{q}$ and baryons are $qqq$. Because of Fermi statistics we need another quantum number, which we call colour. There are three colors
\begin{align*}
\alpha, \beta, \gamma = \text{ red, green, blue},
\end{align*}
and all asymptotic states are colorless. We take color to be the quantum number of strong interactions.
\end{flushleft}

\subsubsection*{QCD Lagrangian}

\begin{flushleft}
Consider the free Lagrangian for quarks of color $\alpha$ and flavor $f$
\begin{align*}
\mathcal{L}_0 = \sum_f \bar{q}_f^{\alpha} (i \cancel{\partial} - m_f)q_f^{\alpha}.
\end{align*}
This Lagrangian is symmetric under global $SU(3)_C$ transformations in color space, and we wish to promote these to local symmetries. The derivative is then no longer invariant, so we must derive a \textit{covariant derivative} which transforms in the same way as $q_f^{\alpha}$ under local transformations. The transformations are given by
\begin{align*}
U = \exp \Big( i \frac{\lambda^a}{2} \theta_a \Big),
\end{align*}
where $\lambda^a$ are the 8 generators of the group. Since this is a non-Abelian group the generators don't commute
\begin{align*}
\Big[ \frac{\lambda^a}{2}, \frac{\lambda^b}{2}
\Big] = i f^{abc} \frac{\lambda^c}{2},
\end{align*}
where $f^{abc}$ are the structure constants. The covariant derivative is
\begin{align*}
D^{\mu} q_f = \Big[ \partial^{\mu} + ig_s \frac{\lambda^a}{2} G^{\mu}_a (x) \Big] qf \equiv [\partial^{\mu} + ig_s G^{\mu} (x)] q_f,
\end{align*}
where $G^{\mu}_a$ are the eight gluons, or gauge bosons, corresponding to the eight generators. Because we want $D^{\mu}$ to transform exactly like the quarks, the transformation of the gluons is fixed
\begin{align*}
D^{\mu} & \rightarrow D^{\mu '} = U D^{\mu} U^{\dagger}\\
G^{\mu} & \rightarrow G^{\mu '} = U G^{\mu} U^{\dagger} + \frac{i}{g_s} (\partial^{\mu} U) U^{\dagger}.
\end{align*}
For infinitesimal $SU(3)_C$ transformations we therefore get
\begin{align*}
q_f^{\alpha} &\rightarrow q_f^{\alpha '} = q_f^{\alpha} + i \Big( \frac{\lambda^a}{2} \Big)_{\alpha \beta} \delta \theta_a q_f^{\beta}\\
G_a^{\mu} & \rightarrow G_a^{\mu '} = G_a^{\mu} - \frac{1}{g_s} \partial^{\mu} (\delta \theta_a) - f^{abc} \delta \theta_b G_c^{\mu}.
\end{align*}
\end{flushleft}

\begin{flushleft}
In order to construct a Lagrangian we must see which terms are allowed. Only terms symmetric under local $SU(3)_c$ transformations are allowed, to there is no mass term e.g.
\begin{align*}
G_{\mu} G^{\mu} &\rightarrow (G_{\mu})'(G^{\mu})' = (U G_{\mu} U^{\dagger} + \frac{i}{g_s} (\partial_{\mu} U) U^{\dagger})\\
& \times (U G^{\mu} U^{\dagger} + \frac{i}{g_s} (\partial^{\mu} U) U^{\dagger})\\
&= U G_{\mu} U^{\dagger} U G^{\mu} U^{\dagger} + \frac{i}{g_s}U G_{\mu} U^{\dagger}  (\partial^{\mu} U)U^{\dagger}\\
&+  \frac{i}{g_s} (\partial_{\mu} U) U^{\dagger}U G^{\mu} U^{\dagger}\\
& - \frac{1}{g_s^2} (\partial_{\mu} U) U^{\dagger} (\partial^{\mu} U) U^{\dagger}\\
&= U G_{\mu}  G^{\mu} U^{\dagger} + \frac{i}{g_s}U G_{\mu} U^{\dagger}  (\partial^{\mu} U)U^{\dagger}\\
&+  \frac{i}{g_s} (\partial_{\mu} U) G^{\mu} U^{\dagger} - \frac{1}{g_s^2} (\partial_{\mu} U) U^{\dagger} (\partial^{\mu} U) U^{\dagger} \neq G_{\mu}G^{\mu}\\
\end{align*}
since $U^{\dagger} U = U^{\dagger} U = 1$. So no mass term is allowed, and the gluons are massless.
\end{flushleft}

\pagebreak

\begin{flushleft}
 We instead consider kinetic terms, analogous to QED
 \end{flushleft}
\end{multicols}
\begin{align*}
i g_s G_{\mu \nu} &= [D_{\mu}, D_{\nu}] = (\partial_{\mu} + ig_s G_{\mu})(\partial_{\nu} + ig_s G_{\nu} )- (\partial_{\nu} + ig_s G_{\nu} ) (\partial_{\mu} + ig_s G_{\mu} )\\
&= \partial_{\mu} \partial_{\nu}+ ig_s \partial_{\mu}  G_{\nu} +ig_s G_{\mu} \partial_{\nu}  - g_s^2 G_{\mu} G_{\nu}
- \partial_{\nu} \partial_{\mu}- ig_s\partial_{\nu}  G_{\mu} -ig_s G_{\nu} \partial_{\mu}  + g_s^2 G_{\nu} G_{\mu}\\
&= [\partial_{\mu}, \partial_{\nu}]+ ig_s (\partial_{\mu}  G_{\nu} - \partial_{\nu}  G_{\mu}) +ig_s [G_{\mu} \partial_{\nu}, G_{\nu} \partial_{\mu}]  + g_s^2 [G_{\mu}, G_{\nu}]\\
&= ig_s (\partial_{\mu}  G_{\nu} - \partial_{\nu}  G_{\mu})   + g_s^2 [G_{\mu}, G_{\nu}]\\
& \rightarrow \frac{\lambda^a}{2} G_a^{\mu \nu} \equiv \partial_{\mu} G_{\nu} - \partial_{\nu} G_{\mu} + g_s [G_{\mu}, G_{\nu}],\\
G_a^{\mu \nu} (x) &=  \partial^{\mu} G^{\nu}_a - \partial^{\nu} G^{\mu}_a - g_s f^{abc} G^{\mu}_b G^{\nu}_c
\end{align*}
\begin{multicols}{2}
\begin{flushleft}
where we've used that
\begin{align*}
[\partial_{\mu}, \partial_{\nu}] &= 0, \text{ } [G_{\mu} \partial_{\nu}, G_{\nu} \partial_{\mu}]=0.\\
\end{align*}
Under $SU(3)_C$ this term transforms in the desired way
\begin{align*}
G^{\mu \nu} \rightarrow G^{\mu \nu '} = U G^{\mu \nu} U^{\dagger}.
\end{align*}
\end{flushleft}

\begin{flushleft}
We now have the terms for the QCD Lagrangian 
\begin{align}
\mathcal{L}_{QCD} \equiv - \frac{1}{4} G^{\mu \nu}_a G^a_{\mu \nu} + \sum_f \bar{q}_f (i \gamma^{\mu} D_{\mu} -m_f)q_f.
\end{align}
We now note a difference between QED and QCD. In QED the gauge bosons commute, and so the field strength only contains terms of te first order in $A_{\mu}$. In QCD, however, the field strength contains a term $G^{\mu}_b G^{\nu}_c$, which means that the Lagrangian has terms that have third and fourth order terms in the gauge boson - meaning that we have self-interaction vertices with three and four gluons, as seen in figure (\ref{QCD vertices}).
 
\begin{figure}[H]
\begin{tikzpicture}
\begin{feynman}
\vertex (a); \vertex [below right=of a] (r); \vertex [above right=of r] (b); \vertex [below left=of r] (c); \vertex [below right=of r] (d); \vertex [left=of r] (D);
\diagram{ (a) --[gluon] (r) -- [gluon] (b); (c)--[gluon] (r) --[gluon] (d)
};
\end{feynman}
\end{tikzpicture}
\begin{tikzpicture}
\begin{feynman}
\vertex (a); \vertex [below right=of a] (r); \vertex [above right=of r] (b); \vertex [below=of r] (c); 
\diagram{ (a) --[gluon] (r) -- [gluon] (b); (c)--[gluon] (r)
};
\end{feynman}
\end{tikzpicture}
\begin{tikzpicture}
\begin{feynman}
\vertex (a); \vertex [below right=of a] (r); \vertex [above right=of r] (b); \vertex [below=of r] (c); 
\diagram{ (a) --[gluon] (r) -- [gluon] (b); (c)--[fermion] (r)
};
\end{feynman}
\end{tikzpicture}
\caption{The interaction vertices from the QCD Lagrangian.}
\label{QCD vertices}
\end{figure}
\end{flushleft}

\subsection*{QCD and QED}
\begin{flushleft}
QCD is a non-Abelian theory, while QED is an Abelian theory. As metnioned in the derivation of $\mathcal{L}_{QCD}$ this means that the generators of $QCD$ don't commute, while the ones in $QED$ do. This leads to the presence of self-interaction vertices in QCD, which could explain asymptotic freedom (antiscreening makes strong interactions weaker at short distances) and confinement (we can't observe free quarks, because strong forces become strong at long distances).
\end{flushleft}


\subsection*{Electroweak unification}
\begin{flushleft}

\end{flushleft}


\end{multicols}

\pagebreak

\section*{Appendix}
\begin{flushleft}
Proof that $[G_{\mu} \partial_{\nu}, G_{\nu} \partial_{\mu}]=0$. Use the identity
\begin{align}
[AB, CD] &= A[B,C]D + [A,C]BD + CA[B,D] + C[A,D]B.
\end{align}
Set in the expression
\begin{align*}
[G_{\mu} \partial_{\nu}, G_{\nu} \partial_{\mu}] &= G_{\mu} [\partial_{\nu}, G_{\nu}] \partial_{\mu} + [G_{\mu}, G_{\nu}] \partial_{\nu}  \partial_{\mu} + G_{\nu} G_{\mu}  [\partial_{\nu}  ,\partial_{\mu}] + G_{\nu}[G_{\mu}, \partial_{\mu}]  \partial_{\nu}\\
&= G_{\mu} [\partial_{\nu}, G_{\nu}] \partial_{\mu} + [G_{\mu}, G_{\nu}] \partial_{\nu}  \partial_{\mu}  + G_{\nu}[G_{\mu}, \partial_{\mu}]  \partial_{\nu} \\
&= G_{\mu} (\partial_{\nu}G_{\nu}-G_{\nu}\partial_{\nu} ) \partial_{\mu} + (G_{\mu} G_{\nu}-G_{\nu} G_{\mu}) \partial_{\nu}  \partial_{\mu}  + G_{\nu}(G_{\mu}\partial_{\mu}-\partial_{\mu}G_{\mu}) \partial_{\nu}\\
&= G_{\mu} \partial_{\nu}G_{\nu} \partial_{\mu}-G_{\mu}G_{\nu}\partial_{\nu}  \partial_{\mu} + G_{\mu} G_{\nu}\partial_{\nu}  \partial_{\mu}-G_{\nu} G_{\mu} \partial_{\nu}  \partial_{\mu}  + G_{\mu}G_{\nu}\partial_{\nu}\partial_{\mu}-G_{\mu}\partial_{\nu}G_{\nu} \partial_{\mu}\\
&= 0\\
\end{align*}
because when indices are summed over they can be interchanged.
\end{flushleft}


\end{document}