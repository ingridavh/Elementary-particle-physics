\documentclass[11pt]{article}
\usepackage{graphicx}
\usepackage[utf8]{inputenc} 
\usepackage{amsmath}
\usepackage{cancel}
\usepackage{bbold}
\usepackage{color}
\usepackage{amsfonts}
\usepackage{mathtools}
\usepackage{braket}
\usepackage{float}
\usepackage{lscape}
\usepackage{multicol}
\usepackage{tikz-feynman}
\usepackage{tikz}
\usepackage{subcaption}
\usepackage{multicol}

\usepackage{geometry}
\geometry{legalpaper, margin=0.5in}

\begin{document}
\title{FYS4560 Project 1}
\author{Ingrid A V Holm}
\maketitle

\begin{multicols}{2}
\section{Standard model and beyond}
\subsection*{- Allowed, forbidden and discovery processes}

\begin{flushleft}
We consider some processes to state whether they are allowed or not. Time direction is left to right in all diagrams. Quantities that should be conserved include but are not limited to
\begin{itemize}
\item Lepton number
\item Baryon number
\item Energy and momentum
\end{itemize}
\end{flushleft}

\subsection*{Electron-positron collisions}

\subsection{$e^+e^- \rightarrow q \bar{q} gg$}
\begin{flushleft}
Can interact through a combination of electroweak and strong interaction, but requires 4 vertices, so the crossection is small
\begin{figure}[H]
\centering
\begin{tikzpicture}
\begin{feynman}[small] 
\vertex (a) {\(e^-\)};
\vertex [below right=of a] (b) ; 
\vertex [below left=of b] (c) {\(e^+\)}; 
\vertex [right=of b] (d); 
\vertex [above right=of d] (e); 
\vertex [above right=of e] (g) {\(q\)}; 
\vertex [below=of g] (p) {\(g\)}; 
\vertex [below right=of d] (f); 
\vertex [below right=of f] (h) {\(\bar{q}\)};
\vertex [above=of h] (p2) {\(g\)};
\diagram{ (a) --[fermion] (b) --[fermion] (c); 
(b) --[photon, edge label={\(\gamma \)}] (d)
 --[fermion] (e) --[fermion] (g);
(e) --[gluon] (p); 
(d) --[fermion] (f) --[fermion] (h); 
(f) --[gluon] (p2)
};
\end{feynman}
\end{tikzpicture}
\begin{tikzpicture}
\begin{feynman}[small] 
\vertex (a) {\(e^-\)};
\vertex [below right=of a] (b) ; 
\vertex [below left=of b] (c) {\(e^+\)}; 
\vertex [right=of b] (d); 
\vertex [above right=of d] (e); 
\vertex [above right=of e] (g) {\(q\)}; 
\vertex [below right=of e] (p) ; 
\vertex [below right=of p] (p1);
\vertex [above right= of p] (p2);
\vertex [below right=of d] (f); 
\vertex [below right=of f] (h) {\(\bar{q}\)};
\diagram{ (a) --[fermion] (b) --[fermion] (c); 
(b) --[photon, edge label={\(\gamma\)}] (d) --[fermion] (e) --[fermion] (g);
(e) --[gluon, edge label={\(g\)}] (p); (d) --[fermion] (h); (p) --[gluon, edge label={\(g\)}] (p1); (p) --[gluon,edge label={\(g\)}] (p2)
};
\end{feynman}
\end{tikzpicture}
\caption{Possible diagrams for $e^+e^- \rightarrow q \bar{q} gg$. The diagram on the left can also have both gluons on one 'arm'.}
\end{figure}
\end{flushleft}
\subsection{$e^+e^- \rightarrow \tilde{l}^+ \tilde{l}^-$}
These are supersymmetric particles, which are very heavy, so this sets kinematic limitations on the process. This decay can happen through the electromagnetic interaction QED. \textit{Allowed.}

\begin{figure}[H]
\centering
\begin{tikzpicture}
\begin{feynman}[small]
\vertex (a) {\(e^-\)}; \vertex [below right=of a] (b); \vertex [below left=of b] (c) {\(e^+\)};
\vertex [right=of b] (d); \vertex [above right=of d] (e) {\( \tilde{l}^+\)}; \vertex [below right=of d] (f) {\( \tilde{l}^-\)};
\diagram{ (a) -- [fermion] (b) --[fermion] (c); (b) --[photon, edge label= {\(\gamma, Z\)}] (d); (f) --[scalar] (d) --[scalar] (e);
};
\end{feynman}
\end{tikzpicture}
\caption{Diagram for $e^+e^- \rightarrow \tilde{l}^- \tilde{l}^+$.}
\end{figure}

\subsection{$e^+ e^- \rightarrow HH \gamma$}
\begin{flushleft}
Allowed through the weak and electroweak interaction. There are three $5$-vertex diagrams, but two of them include a $WW \gamma$-vertex, which has a second order coupling constant.
\begin{figure}[H]
\centering
\begin{tikzpicture}
\begin{feynman}[small]
\vertex (r1) {\(e^-\)}; \vertex [below right=of r1] (r2); \vertex [below left=of r2] (r3) {\(e^+\)}; \vertex [right=of r2] (p); \vertex [above right=of p] (p1); \vertex [below right= of p] (p2); \vertex [above right=of p2] (k); \vertex [right=of k] (m);
\vertex [above=of m] (t) {\(H^0\)}; \vertex [below=of m] (b) {\(H^0\)};
\diagram{ (r1) --[fermion] (r2) --[fermion](r3); (r2) --[photon, edge label={\(Z, \gamma\)}] (p); (p) --[fermion, edge label={\(t\)}] (p1) --[fermion, edge label={\(t\)}] (k) --[fermion, edge label={\(t\)}] (p2) --[fermion, edge label={\(t\)}] (p); (p1) --[scalar] (t); (p2) --[scalar] (b); (k) --[photon, edge label={\(\gamma\)}] (m)
};
\end{feynman}
\end{tikzpicture}
\caption{Diagram for $e^+ e^- \rightarrow HH \gamma$.}
\end{figure}
\end{flushleft}

\subsection{$e^+ e^- \rightarrow ZZZ$}
\begin{flushleft}
Important because it's a precision test of vertices containing to $Z$ bosons. Since this process creates two jets; one where two $Z$'s come out, and one where one comes out, we see that $ZZZ$-vertices are not observed. It could also occur via the electroweak interaction, but then the resulting bosons would be $\gamma ZZ$ or $\gamma \gamma \gamma$.
\begin{figure}[H]
\centering
\begin{tikzpicture}
\begin{feynman}[small]
\vertex (r1) {\(e^-\)}; \vertex [below right=of r1] (r2); \vertex [below left=of r2] (r3) {\(e^+\)}; \vertex [right= of r2] (p); \vertex [below right=of p] (p1); \vertex [above right = of p1] (p11) {\(Z\)}; \vertex [below right= of p1] (p12) {\(Z\)}; \vertex [above=of p11] (p2) {\(Z\)};
\diagram{(r1) --[fermion] (r2) --[fermion] (r3); (r2) --[photon, edge label={\(Z\)}] (p); (p) --[scalar, half left, edge label={\(W\)}] (p1) --[scalar, half left, edge label={\(W\)}] (p); (p11) --[photon] (p1) --[photon] (p12); (p) --[photon] (p2)
};
\end{feynman}
\end{tikzpicture}
\caption{Diagram for $e^+ e^- \rightarrow ZZZ$.}
\end{figure}
\end{flushleft}
\subsection{$e^+e^- \rightarrow H \rightarrow gg$}
This process is allowed via the strong and the weak interaction, and produces a Higgs boson which then decays into top quark production.
\begin{figure}[H]
\centering
\begin{tikzpicture}
\begin{feynman}[small]
\vertex (r1) {\(e^-\)}; \vertex[right =of r1] (p1); \vertex[below right=of p1] (k); \vertex [below left=of k] (p2); \vertex [left=of p2] (r2) {\(e^+\)};  \vertex [right=of k] (m); \vertex[above right = of m] (m1); \vertex [below right=of m] (m2); \vertex [right = of m1] (m11) {\(g\)}; \vertex [right=of m2] (m22) {\(g\)};
\diagram{
(r1) --[fermion] (p1) --[fermion] (p2) --[fermion] (r2); 
(p1) --[photon, edge label={\(Z, \gamma\)}] (k) --[photon, edge label={\(Z, \gamma\)}] (p2); 
(k) --[scalar, edge label={\(H^0\)}] (m);
 (m) --[fermion, edge label={\(t\)}] (m1) --[fermion, edge label={\(t\)}] (m2) --[fermion, edge label={\(t\)}] (m);
(m1) --[gluon] (m11); (m2) --[gluon] (m22) 
};
\end{feynman}
\end{tikzpicture}
\caption{Diagram for $e^-e^+ \rightarrow H \rightarrow gg$.}
\end{figure}
\begin{flushleft}

\end{flushleft}

\subsection{$e^+e^- \rightarrow \nu \bar{\nu} \gamma \gamma$}
\begin{flushleft}
This process is allowed through the electroweak interaction.
\begin{figure}[H]
\begin{tikzpicture}
\begin{feynman}[small]
\vertex (r1) {\(e^-\)}; \vertex [below=of r1] (r2) {\(e^+\)}; \vertex [right = of r1] (p1); \vertex [right=of r2] (p2);
\vertex [right =of p2] (k2); \vertex[right =of p1] (k1);
\vertex [right=of k1] (m1) {\(\nu\)}; \vertex [above =of m1] (m11) {\(\gamma\)}; \vertex [right= of k2] (m2) {\(\bar{\nu}\)}; \vertex [below=of m2] (m22) {\(\gamma\)};
\diagram{ (r1) --[fermion] (p1) --[fermion] (k1) --[fermion] (m1); (m2) --[fermion] (k2) --[fermion] (p2) --[fermion] (r2);
(p1) --[photon] (m11); (p2) --[photon] (m22); (k1) --[scalar, edge label={\(W\)}] (k2)
};
\end{feynman}
\end{tikzpicture}
\begin{tikzpicture}
\begin{feynman}[small]
\vertex (r1) {\(e^-\)}; 
\vertex [right =of r1] (p1); 
\vertex [below right=of p1] (k); 
\vertex [below left= of k] (p2); 
\vertex [left=of p2] (r2) {\(e^+\)};
\vertex [above right=of k] (k1) {\(\gamma\)}; 
\vertex [below right= of k] (k2) {\(\gamma\)}; 
\vertex [above= of k1] (k11) {\(\nu\)}; 
\vertex [below=of k2] (k22) {\(\bar{\nu}\)};
\diagram{(r1) --[fermion] (p1) --[fermion] (k11); 
(k22) --[fermion] (p2) --[fermion] (r2);
(p1) --[scalar, edge label={\(W\)}] (k) --[photon] (k1); (k2)--[photon] (k)--[scalar, edge label={\(W\)}] (p2)
};
\end{feynman}
\end{tikzpicture}
\caption{Diagrams for $e^+e^- \rightarrow \nu \bar{\nu} \gamma \gamma$.}
\end{figure}
\end{flushleft}

\subsection{$e^+e^- \rightarrow Y(3s) \rightarrow B^0 \bar{B}^0$}
\begin{flushleft}
This process is allowed through the electroweak interaction, or a combination of the electroweak and strong interaction.
\begin{figure}[H]
\centering
\begin{tikzpicture}
\begin{feynman}[small]
\vertex (r1); \vertex [below right= of r1] (r); \vertex [below left=of r] (r2); \vertex [right=of r] (p); 
\vertex [right of= p] (k); \vertex [right of= k] (k2); \vertex [above right=of k2] (k21) {\(d\)}; \vertex [below=of k21] (k22) {\(\bar{d}\)}; \vertex [above=of k21] (k1) {\(\bar{b}\)}; \vertex [below=of k22] (k3) {\(b\)};
\diagram{ (r1) --[fermion] (r)--[fermion] (r2); (r) --[photon, edge label={\(\gamma, Z\)}] (p); (k1) --[fermion] (k) --[fermion] (p) --[fermion] (k3); (k2) --[photon, edge label={\(\gamma, Z, g\)}] (k); (k22) --[fermion] (k2) --[fermion] (k21)
};
\end{feynman}
\end{tikzpicture}
\caption{Diagram for $e^+e^- \rightarrow Y(3s) \rightarrow B^0 \bar{B}^0$. $Y(3s)$ is made up of $\bar{b}b $.}
\end{figure}
\end{flushleft}

\subsection{$e^+ e^- \rightarrow Z^0 t \bar{t}$}
\begin{flushleft}
This process is allowed through the electroweak interaction, and through the production of a Higgs boson. Top-quark pairs are normally produced through a virtual gluon, but can also be produced from a virtual photon or Z boson. The Higgs can also decay into $t \bar{t}$, but as this process requires a very high energy it's more likely to decay into $b \bar{b}$. We have, however, included both diagrams here.
\begin{figure}[H]
\centering
\begin{tikzpicture}
\begin{feynman}[small]
\vertex (r1) {\(e^-\)}; \vertex [right=of r1] (p1); \vertex [below=of p1](p2); \vertex [left=of p2] (r2) {\(e^+\)}; \vertex [right=of p1] (k); \vertex [above right= of k] (k1) {\(t\)}; \vertex [below right= of k] (k2) {\(\bar{t}\)}; \vertex [below=of k2] (k3) {\(Z\)};
\diagram{ (r1) --[fermion] (p1) --[fermion] (p2) --[fermion] (r2); (p1) --[photon, edge label={\(\gamma, Z\)}] (k); (k2) --[fermion] (k) --[fermion] (k1); (p2) --[photon] (k3) 
};
\end{feynman}
\end{tikzpicture}
\begin{tikzpicture}
\begin{feynman}[small]
\vertex (r1) {\(e^-\)}; \vertex [below right=of r1] (r); \vertex [below left=of r] (r2) {\(e^+\)}; \vertex [right=of r] (p); \vertex[right =of p] (p1); \vertex [right=of p1] (p12) {\(\bar{t}\)}; \vertex [above=of p12] (p11){\(t\)}; \vertex [below=of p12] (p2) {\(Z\)}; 
\diagram{ (r1) --[fermion] (r) --[fermion] (r2); (r) --[photon, edge label={\(Z\)}] (p) --[photon] (p2); (p12) --[fermion] (p1) --[fermion] (p11); (p) --[scalar, edge label={\(H^0\)}] (p1)
};
\end{feynman}
\end{tikzpicture}
\caption{Diagrams for $e^+ e^- \rightarrow Z^0 t \bar{t}$.}
\end{figure}
\end{flushleft}

\subsection*{Gluon-gluon fusion}

\subsection{$gg \rightarrow e^+e^-$}
\begin{flushleft}
This interaction is possible, but requires four vertices so its cross section must be small. It is a combination of the electroweak and strong interaction.
\end{flushleft}
\begin{figure}[H]
\centering
\begin{tikzpicture}
\begin{feynman}[small]
\vertex (a) {\(g\)}; \vertex [below right=of a] (b); \vertex [below left=of b] (c) {\(g\)}; \vertex [right=of b] (d); \vertex [right= of d] (e);
\vertex [right= of e] (f); \vertex [above right=of f] (g) {\(e^-\)}; 
\vertex [below right= of f ] (h) {\(e^+\)};
\diagram{ (a) --[gluon] (b) --[gluon] (c); (b) -- [gluon] (d); (d) --[fermion, half left, edge label={\(q\)}] (e) --[fermion, half left, edge label={\(q\)}] (d);(e) --[photon, edge label={\(\gamma, Z\)}] (f); (h) --[fermion] (f) --[fermion] (g)
};
\end{feynman}
\end{tikzpicture}
\begin{tikzpicture}
\begin{feynman}[small]
\vertex (r); \vertex [above left=of r] (r1); \vertex [left=of r1] (r11) {\(g\)}; \vertex [below left=of r] (r2); \vertex [left=of r2] (r22) {\(g\)};
\vertex [right = of r] (p); \vertex [above right=of p] (p1) {\(e^-\)}; \vertex [below right=of p] (p2) {\(e^+\)};
\diagram{ (r) --[fermion, edge label={\(q\)}] (r2) --[fermion, edge label={\(q\)}] (r1) --[fermion, edge label={\(q\)}] (r); (r1) --[gluon] (r11); (r2) --[gluon] (r22); (r) --[photon, edge label={\(\gamma, Z\)}] (p); (p2) --[fermion] (p) --[fermion] (p1)
};
\end{feynman}
\end{tikzpicture}
\caption{Diagrams for $gg \rightarrow e^- e^+$.}
\end{figure}

\subsection{$gg \rightarrow t \bar{t} HH$}
\begin{flushleft}
This process is allowed via the strong interaction. It's important in order to determine whether there are $HH$-vertices.
\begin{figure}[H]
\centering
\begin{tikzpicture}
\begin{feynman}[small]
\vertex (r1) {\(g\)}; \vertex [right=of r1] (p1) ; \vertex [below=of p1] (p2) ; \vertex [left=of p2] (r2) {\(g\)};
\vertex [right=of p1] (k1); \vertex [above right=of k1] (k11) {\(\bar{t}\)};\vertex [below= of k11] (k12) {\(H^0\)}; \vertex [right= of p2] (k2); \vertex [below right= of k2] (k22) {\(t\)}; \vertex [above= of k22] (k21) {\(H^0\)};
\diagram{ (r1) --[gluon] (p1); (r2) --[gluon] (p2); (k11) --[fermion] (k1) --[fermion, edge label={\(t\)}] (p1) --[fermion, edge label={\(t\)}] (p2) -- [fermion, edge label={\(t\)}] (k2) --[fermion] (k22); (k2) --[scalar] (k21); (k1) --[scalar] (k12)
};
\end{feynman}
\end{tikzpicture}
\caption{Diagram for $gg \rightarrow t \bar{t} HH$.}
\end{figure}
\end{flushleft}

\subsection{$gg \rightarrow H \rightarrow Z \gamma$}
\begin{flushleft}
This process is allowed through a combination of the strong and electroweak interaction, using a top quark loop. This diagram is important for the top quark production and decay.
\begin{figure}[H]
\begin{tikzpicture}
\begin{feynman}
\vertex (r1) {\(g\)}; \vertex [right=of r1] (p1) ; \vertex [below=of p1] (p2) ; \vertex [left=of p2] (r2) {\(g\)};
\vertex [below right= of p1] (p); \vertex [right=of p] (k); \vertex [above right=of k] (k1); \vertex [below=of k1] (k2); \vertex [right =of k1] (k11) {\(Z, \gamma\)}; \vertex [right=of k2] (k22) {\(\gamma, Z\)};
\diagram{ (r1) --[gluon] (p1) --[fermion, edge label={\(t\)}] (p) --[fermion, edge label={\(t\)}] (p2) --[fermion, edge label={\(t\)}] (p1); (r2) --[gluon] (p2); (p) --[scalar, edge label={\(H^0\)}] (k); (k) --[scalar, edge label={\(W\)}] (k1) --[scalar, edge label={\(W\)}] (k2) --[scalar, edge label={\(W\)}] (k); (k1) --[photon] (k11); (k2) --[photon] (k22)
};
\end{feynman}
\end{tikzpicture}
\caption{Diagrams for $gg \rightarrow H \rightarrow Z \gamma$.}
\end{figure}
\end{flushleft}


\subsection*{Quark-antiquark collisions}

\subsection{$q \bar{q} \rightarrow W^+ W^- Z$}
\begin{flushleft}
This process is allowed through the weak interaction. It is important because it gives evidence for the existence of the $WWZZ$-four vertex.
\begin{figure}[H]
\centering
\begin{tikzpicture}
\begin{feynman}[small]
\vertex (r1) {\(q\)}; \vertex [below right=of r1] (r2); \vertex [below left=of r2] (r3) {\(\bar{q}\)}; \vertex[right=of r2] (p); \vertex [right =of p] (k2) {\(Z\)}; \vertex [above=of k2] (k1) {\(W\)}; \vertex [below=of k2] (k3) {\(W\)};
\diagram{(r1) --[fermion] (r2) --[fermion] (r3); (r2) --[photon, edge label={\(Z\)}] (p); (k1) --[scalar] (p) --[scalar] (k3); (p) --[photon] (k2)
};
\end{feynman}
\end{tikzpicture}
\end{figure}
\end{flushleft}


\subsection{$q \bar{q} \rightarrow gg e^+ e^-$}
\begin{flushleft}
This process is allowed through a combination of the strong and electroweak interactions.
\begin{figure}[H]
\centering
\begin{tikzpicture}
\begin{feynman}[small]
\vertex (r1) {\(q\)}; \vertex [right= of r1] (p1); \vertex[below right= of p1](p); \vertex [below left= of p] (p2); \vertex [left = of p2] (r2) {\(\bar{q}\)}; \vertex [right= of p] (k); \vertex [above right=of k] (k1) {\(e^-\)}; \vertex [below right= of k] (k2) {\(e^+\)}; \vertex[above= of k1] (k0) {\(g\)} ;\vertex [below= of k2] (k3) {\(g\)};
\diagram{ (r1) --[fermion] (p1) --[fermion] (p) --[fermion] (p2) --[fermion] (r2); (p1)--[gluon] (k0); (p2) --[gluon] (k3); (k2) --[fermion] (k) --[fermion] (k1); (p)--[photon, edge label={\(\gamma, Z\)}] (k)
};
\end{feynman}
\end{tikzpicture}
\begin{tikzpicture}
\begin{feynman}[small]
\vertex (r1) {\(q\)}; \vertex [right=of r1] (p1); \vertex [below=of p1] (p2); \vertex [left=of p2] (r2) {\(\bar{q}\)}; \vertex [right=of p1] (k1); \vertex[above right=of k1] (k11) {\(g\)}; \vertex [below =of k11] (k12) {\(g\)}; \vertex [right =of p2] (k2); \vertex [below right=of k2] (k22) {\(e^+\)}; \vertex [above=of k22] (k21) {\(e^-\)};
\diagram{ (r1) --[fermion] (p1) --[fermion] (p2) --[fermion] (r2); (p1) --[gluon] (k1) --[gluon] (k11); (k1) --[gluon] (k12); (p2) --[photon, edge label={\(\gamma, Z\)}] (k2); (k22) --[fermion] (k2)--[fermion](k21)
};
\end{feynman}
\end{tikzpicture}
\end{figure}
\end{flushleft}


\subsection*{Proton-proton and proton-antiproton collisions}

\subsection{$p \bar{p} \rightarrow l^+ l^- X$}
\begin{flushleft}
Quark and antiquark from proton and antiproton can annihilate and become lepton and antilepton through a virtual photon. \textit{Allowed.}
\begin{figure}[H]
\centering
\begin{tikzpicture}
\begin{feynman}[small]
\vertex (r); \vertex [above left=of r] (r1) {\(q\)}; \vertex [below left=of r] (r2) {\(\bar{q}\)}; \vertex [right= of r] (p); \vertex [above right=of p] (p1) {\(l^-\)}; \vertex [below right= of p] (p2) {\(l^+\)};
\diagram{ (r1) --[fermion] (r) --[fermion] (r2); (r) --[photon, edge label={\(\gamma, Z\)}] (p); (p2) --[fermion] (p) --[fermion] (p1)
};
\end{feynman}
\end{tikzpicture}
\caption{Diagram for $p \bar{p} \rightarrow l^+ l^-$.}
\end{figure}
\end{flushleft}

\subsection{$pp \rightarrow l^+ l^- l^+ l^- X$}
\begin{flushleft}
This process is important because it requires the production of a Higgs boson. It is allowed via the weak interaction.
\begin{figure}[H]
\centering
\begin{tikzpicture}
\begin{feynman}[small]
\vertex (r1) {\(q\)}; \vertex [right=of r1] (p1);
\vertex [below right= of p1] (p2); 
\vertex [below left=of p2] (p3); 
\vertex [left=of p3] (r3) {\(q\)}; 
\vertex [right= of p2] (k);
\vertex [above right = of k] (k1);
\vertex [above right = of k1] (k11) {\(l^-\)};
\vertex [below= of k11] (k12) {\(l^+\)};
\vertex [below right = of k] (k2);
\vertex [above right = of k2] (k21) {\(l^-\)};
\vertex [below= of k21] (k22) {\(l^+\)};
\vertex [above = of k11] (t) {\(q\)};
\vertex [below = of k22] (b) {\(q\)};
\diagram{ (r1) --[fermion] (p1) --[fermion] (t); (r3) --[fermion] (p3) --[fermion] (b); (p1) -- [photon, edge label={\(Z\)}](p2) --[photon, edge label={\(Z\)}] (p3); (p2) --[scalar, edge label={\(H^0\)}] (k); (k2) --[photon, edge label={\(Z\)}] (k) --[photon, edge label={\(Z\)}] (k1); (k12) --[fermion] (k1) --[fermion] (k11); (k22) --[fermion] (k2) --[fermion] (k21)
};
\end{feynman}
\end{tikzpicture}
\caption{Diagram for $p \bar{p} \rightarrow l^+ l^-$.}
\end{figure}
\end{flushleft}

\subsection*{Collisions}

\subsection{$K^- p \rightarrow \Omega^- K^+ K^0$}
\begin{flushleft}
This interaction is allowed through the strong interaction.
\begin{figure}[H]
\centering
\begin{tikzpicture}
\begin{feynman}[small]
\vertex (r); 
\vertex [above left= of r] (r1) {\(u\)}; 
\vertex [below left= of r] (r2) {\(\bar{u}\)}; 
\vertex [right=of r] (p); 
\vertex [above right=of p] (p1) {\(s\)}; 
\vertex [below right=of p] (p2) {\(\bar{s}\)};
\vertex [below=of r2] (m) {\(d\)}; 
\vertex [right=of m] (n); 
\vertex [below= of p2] (n2) {\(d\)};
\vertex [below=of n] (o);  
\vertex [below =of n2] (o2) {\(\bar{s}\)};
\vertex [below =of o2] (o1) {\(s\)};
\diagram{ (r2) --[fermion] (r) --[fermion] (r1); (r) --[gluon, edge label={\(g, \gamma\)}] (p); (p2) --[fermion] (p) --[fermion] (p1); (m) --[fermion] (n) --[fermion] (n2); (n) --[gluon, edge label={\(g, \gamma\)}] (o); (o2) --[fermion] (o) --[fermion] (o1)
};
\end{feynman}
\end{tikzpicture}
\begin{tikzpicture}
\begin{feynman}[small]
\vertex [below=of r2](R); 
\vertex [above left=of R] (R1) {\(\bar{u}\)}; 
\vertex [below left=of R] (R2) {\(u\)}; 
\vertex [right=of R] (P); 
\vertex [above right=of P] (P1); 
\vertex [below right=of P] (P2); 
\vertex [above right = of P1] (P11) {\(\bar{s}\)}; 
\vertex [below=of P11] (P12) {\(s\)}; 
\vertex [below right=of P2] (P22) {\(s\)}; 
\vertex [above=of P22] (P21) {\(\bar{s}\)};
\diagram{(R2) --[fermion] (R) --[fermion] (R1); 
(R) --[gluon, edge label={\(g\)}] (P); 
(P1) --[gluon, edge label={\(g\)}] (P) --[gluon, edge label={\(g\)}] (P2); 
(P11) --[fermion] (P1) --[fermion] (P12);
(P21) --[fermion] (P2) --[fermion] (P22)};
\end{feynman}
\end{tikzpicture}
\caption{Diagrams for $K^- p \rightarrow \Omega^- K^+ K^0$.}
\end{figure}
\end{flushleft}

\subsection{$\nu_e e^- \rightarrow \nu_e e^- \gamma$}
\begin{flushleft}
Allowed through electroweak interaction.
\begin{figure}[H]
\begin{tikzpicture}
\begin{feynman}
\vertex (r1) {\(e^-\)}; 
\vertex [right= of r] (p1); 
\vertex [below =of p1] (p2); 
\vertex [below= of p2] (p3); 
\vertex [left=of p3] (r3) {\(\nu_e\)}; 
\vertex [right =of p2] (l); 
\vertex [right= of l] (k2); 
\vertex [above=of k2] (k1) {\(e^-\)}; 
\vertex [below =of k2] (k3) {\(\nu_e\)};
\diagram{ (r1) --[fermion] (p1) --[fermion] (k1); (p3) -- [photon, edge label={\(Z,\gamma\)}] (p2) --[photon, edge label={\(Z, \gamma\)}] (p1); (r3) --[fermion] (p3) --[fermion] (k3); (p2) --[scalar, half left, edge label={\(W\)}] (l) --[scalar, half left, edge label={\(W\)}] (p2); (l) --[photon, edge label={\(\gamma\)}] (k2)
};
\end{feynman}
\end{tikzpicture}
\caption{Diagram for $\nu_e e^- \rightarrow \nu_e e^- \gamma$.}
\end{figure}
\end{flushleft}

\subsection{$ep \rightarrow J/ \psi +X$}
\begin{flushleft}
The $J/ \psi$ meson consists of $c \bar{c}$ quarks, a proton consists of $uud$ and a neutron consists of $udd$. So in order for the process to occur , an up quark must turn into a down quark. This process is allowed through the weak interaction. It's important because it shows that flavour is not conserved in weak interactions. 
\begin{figure}[H]
\centering
\begin{tikzpicture}
\begin{feynman}[small]
\vertex (r1) {\(u\)}; \vertex [right =of r1] (p1); \vertex[below right= of p1] (p); \vertex [below left= of p] (p2); \vertex [left=of p2] (r2) {\(e^-\)}; \vertex [right=of p] (k); \vertex[above right = of k] (k1) {\(c\)}; \vertex [above=of k1] (k0) {\(d\)}; \vertex[below right=of k] (k2) {\(\bar{c}\)}; \vertex [below= of k2] (k3) {\(\nu_e\)};
\diagram{ (r1) --[fermion] (p1) --[scalar, edge label={\(W\)}] (p) --[scalar, edge label={\(W\)}] (p2) --[fermion] (r2); (p1) --[fermion] (k0); (p2) --[fermion] (k3); (p) --[photon, edge label={\(Z\)}] (k); (k2) --[fermion] (k) --[fermion] (k1)
};
\end{feynman}
\end{tikzpicture}
\caption{Diagram for $ep \rightarrow J/ \psi +X$.}
\end{figure}
\end{flushleft}


\subsection*{Decays}

\subsection{$\tau^+ \rightarrow \mu^+ \nu_e \bar{\nu}_{\tau}$}
\begin{flushleft}
This process is not allowed because it violates lepton flavour conservation. If we changed $\nu_e$ by $\nu_{\mu}$ it would be allowed.
\end{flushleft}

\subsection{$D^0 \leftrightarrow \bar{D}^0$}
\begin{flushleft}
This process is a neutral current, where the weak interaction changes the quark flavour. The mean lifetime of $D^0$ and $\bar{D}^0$ is $(4.101 \pm 0.015) \times 10^{-13}$ s.
\begin{figure}[H]
\centering
\begin{tikzpicture}
\begin{feynman}
\vertex (r1) {\(\bar{u}\)}; 
\vertex [below=of r1] (r2) {\(c	\)};
 \vertex [right=of r1] (p1); 
 \vertex [right=of p1] (k1); 
 \vertex[right=of k1] (m1) {\(\bar{c}\)}; 
 \vertex [right=of r2] (p2); 
 \vertex [right=of p2] (k2); 
 \vertex [right=of k2] (m2) {\(u\)}; 
\diagram{ (m1) --[fermion] (k1) --[fermion, edge label={\(d, s, b\)}] (p1) --[fermion] (r1); (r2) --[fermion] (p2) --[fermion, edge label={\(d, s, b\)}] (k2) --[fermion] (m2); 
(p1) --[scalar, edge label={\(W\)}] (p2); (k1) --[scalar, edge label={\(W\)}] (k2)
};
\draw [decoration ={brace}, decorate] (r2.south west)-- (r1.north west) node [pos=0.5, left] {\(D^{0}\)};
\draw [decoration ={brace}, decorate] (m1.north east)-- (m2.south east) node [pos=0.5, right] {\(\bar{D}^{0}\)};
\end{feynman}
\end{tikzpicture}
\begin{tikzpicture}
\begin{feynman}
\vertex (r1) {\(\bar{u}\)}; 
\vertex [below=of r1] (r2) {\(c	\)};
 \vertex [right=of r1] (p1); 
 \vertex [right=of p1] (k1); 
 \vertex[right=of k1] (m1) {\(\bar{c}\)}; 
 \vertex [right=of r2] (p2); 
 \vertex [right=of p2] (k2); 
 \vertex [right=of k2] (m2) {\(u\)}; 
\diagram{ (r2) --[fermion] (p2) --[fermion, edge label={\(d, s, b\)}] (p1) --[fermion] (r1); (m1) --[fermion] (k1) --[fermion, edge label={\(d, s, b\)}] (k2) --[fermion] (m2); 
(p1) --[scalar, edge label={\(W\)}] (k1); (p2) --[scalar, edge label={\(W\)}] (k2)
};
\draw [decoration ={brace}, decorate] (r2.south west)-- (r1.north west) node [pos=0.5, left] {\(D^{0}\)};
\draw [decoration ={brace}, decorate] (m1.north east)-- (m2.south east) node [pos=0.5, right] {\(\bar{D}^{0}\)};
\end{feynman}
\end{tikzpicture}
\caption{Diagrams for $D^0 \leftrightarrow \bar{D}^0$.}
\end{figure}
\end{flushleft}

\pagebreak


%-----------START DEL 2


\section{Top quark and $W$-boson}

\subsection*{CKM-matrix}
\begin{flushleft}
The electroweak interaction does not preserve quark flavour. Quarks can change flavour by emitting a $W^{\pm}$-boson. The flavour-changes occur according to the classification of quarks into upper and lower, and the doublets
\begin{align*}
\begin{pmatrix}
u\\
d'\\
\end{pmatrix},
\begin{pmatrix}
c\\
s'\\
\end{pmatrix},
\begin{pmatrix}
t\\
b'\\
\end{pmatrix},
\end{align*}
where $d'$,$s'$ and $b'$ are the weak eigenstates constructed from the quarks $d,s,b$
\begin{align*}
\begin{pmatrix}
d'\\
s'\\
b'\\
\end{pmatrix} = 
V_{CKM} \begin{pmatrix}
d\\
s\\
b\\
\end{pmatrix}.
\end{align*}
The matrix $V_{CKM}$ relates the weak interaction eigenstates $d'$ to the energy eigenstates $d$, and is called the \textit{Cabbibo-Kobayashi-Maskawa matrix}. The current best approximation is given by $V_{CKM}$
\begin{align*}
\begin{pmatrix}
V_{ud} & V_{us} & V_{ub}\\
V_{cd} & V_{cs} & V_{cb}\\
V_{td} & V_{ts} & V_{tb}\\
\end{pmatrix}
\simeq \begin{pmatrix}
0.97427 & 0.22534 & 0.00351 \\
0.22520 & 0.97344 & 0.0412 \\
0.00867 & 0.0400 & 0.999146\\
\end{pmatrix}
\end{align*}
\begin{figure}[H]
\begin{tikzpicture}
\begin{feynman}
\vertex (a) {\(q_u\)}; \vertex [right=of a] (b); \vertex [above right=of b] (c); \vertex [below right=of b] (d) {\(q_d\)};
\diagram{(a)--[fermion] (b) --[photon, edge label={\(W^{\pm}\)}] (c); (b) --[fermion] (d)
};
\end{feynman}
\end{tikzpicture}
\centering
\label{fig:: flavour change}
\caption{Flavour-changing weak interaction with quarks.}
\end{figure}
The interactions of quarks and leptons with the $W^{\pm}$ bosons are called charged currents. Parity and charge conjugation are broken, but $CP$ is usually conserved. The top quark mass $m_t = 173$ GeV, however, is so large that it's production with the bottom quark
\begin{align*}
W^- \rightarrow t b',
\end{align*}
is kinematically forbidden. We will consider some ways the top quark is produced.
\end{flushleft}

\subsection*{$B^0-\bar{B}^0$-oscillations}
\begin{flushleft}
A manifestation of the neutral particle oscillations and charged current, is the $B^0-\bar{B}^0$ oscillations between particle and antiparticle. The neutral $B^0$ meson consists of a strange antiquark $s$ and a bottom quark $b$, or their antiparticles. The mixing has been observed at Fermilab in 2006 and by LHCb at CERN in 2011. The oscillation is important because it can tell us something about the excess of matter (as opposed to antimatter) in the universe. The $V_{CKM}$-matrix elements that go into the oscillations are $V_{tb}$ and $V_{ts}$ for the $B_s$ meson, and $V_{tb}$ and $V_{td}$ for the $B$ meson, as can be seen in Fig. (\ref{fig:: b-bbar mixing}).
\begin{figure}[H]
\centering
\begin{tikzpicture}
\begin{feynman}
\vertex (r1) {\(\bar{b}\)}; 
\vertex [below=of r1] (r2) {\(d\)}; 
\vertex [right=of r1] (p1); 
\vertex [below=of p1] (p2); 
\vertex [right=of p1] (k1);
\vertex [below=of k1] (k2);
\vertex [right=of k1] (q1) {\(\bar{d}\)};
\vertex [below=of q1] (q2) {\(b\)};
\diagram{ (r2) --[fermion] (p2) --[fermion, edge label={\((u,c),t\)}] (p1) --[fermion] (r1); (p1) --[scalar, edge label={\(W\)}] (k1); (p2) --[scalar, edge label={\(W\)}] (k2); (q1) --[fermion] (k1) --[fermion, edge label={\((u,c),t\)}] (k2) --[fermion] (q2)
};
\draw [decoration ={brace}, decorate] (r2.south west)-- (r1.north west) node [pos=0.5, left] {\(B^{0}\)};
\draw [decoration ={brace}, decorate] (q1.north east)-- (q2.south east) node [pos=0.5, right] {\(\bar{B}^{0}\)};
\end{feynman}
\end{tikzpicture}
\begin{tikzpicture}
\begin{feynman}
\vertex (r1) {\(\bar{b}\)}; 
\vertex [below=of r1] (r2) {\(s\)}; 
\vertex [right=of r1] (p1); 
\vertex [below=of p1] (p2); 
\vertex [right=of p1] (k1);
\vertex [below=of k1] (k2);
\vertex [right=of k1] (q1) {\(\bar{s}\)};
\vertex [below=of q1] (q2) {\(b\)};
\diagram{ (r2) --[fermion] (p2) --[fermion, edge label={\((u,c),t\)}] (p1) --[fermion] (r1); (p1) --[scalar, edge label={\(W\)}] (k1); (p2) --[scalar, edge label={\(W\)}] (k2); (q1) --[fermion] (k1) --[fermion, edge label={\((u,c),t\)}] (k2) --[fermion] (q2)
};
\draw [decoration ={brace}, decorate] (r2.south west)-- (r1.north west) node [pos=0.5, left] {\(B^{0}_s\)};
\draw [decoration ={brace}, decorate] (q1.north east)-- (q2.south east) node [pos=0.5, right] {\(\bar{B}^{0}_s\)};
\end{feynman}
\end{tikzpicture}
\caption{Diagrams for $B^0-\bar{B}^0$ and $B^0_s-\bar{B}^0_s$ oscillations.}
\label{fig:: b-bbar mixing}
\end{figure}
\end{flushleft}

\subsection*{Top quark}

\begin{flushleft}
The top quark is the heaviest of the quarks. It interacts mainly through the strong force, but can only decay via the weak interaction as this is the only interaction that doesn't conserve favour.
\end{flushleft}

\subsubsection*{Production}
\begin{flushleft}
The top quark can be produced either as part of a \textit{top-quark pair} or \textit{single top quark}. Because of the large top quark mass, these quarks must be produced in high energy collision, e.g. proton-proton, proton-antiproton and electron-positron collisions at a center of mass energy of $7$ TeV ($m_t \simeq 172$ GeV). In proton-antiproton collisions, a quark-antiquark pair can annihilate into a virtual gluon through the strong interaction, which subsequently decays into a top-antitop pair, as seen in Fig. (\ref{fig:: top pair gluon}).  
\begin{figure}[H]
\centering
\begin{tikzpicture}
\begin{feynman}
\vertex (r1) {\(q\)}; \vertex [below right= of r1] (r); \vertex [below left=of r] (r2) {\(\bar{q}\)}; \vertex [right=of r] (p); \vertex [above right=of p] (p1) {\(t\)}; \vertex [below right=of p] (p2) {\(\bar{t}\)};
\diagram{ (r1) --[fermion] (r) --[fermion] (r2); (r) --[gluon, edge label={\(g\)}] (p); (p2) --[fermion] (p) --[fermion] (p1)
};
\end{feynman}
\end{tikzpicture}
\caption{Feynman diagram for top-quark pair production from gluon in $p \bar{p}$-collision.}
\label{fig:: top pair gluon}
\end{figure}
Top-quark pairs can also be produced from a gluon pair, or from a single virtual gluon exchanged between two quarks in a proton-proton collision, as seen in Fig. (\ref{fig:: top pair pp}). A much rarer production channel is through the electroweak interaction, which could happen in electron-positron collisions. The signature would be the same as in a hadron collider.
\begin{figure}[H]
\centering
\begin{tikzpicture}
\begin{feynman}[small]
\vertex (r1) {\(q\)}; \vertex [right=of r1] (p1); 
\vertex [below=of p1] (p); \vertex [below =of p] (p2); \vertex [left=of p2] (r2) {\(q\)}; \vertex [right=of p] (k); \vertex [right=of p1] (k1); \vertex [right =of k1] (m1) {\(q\)}; \vertex [right=of p2] (k2) ;\vertex [right of=k2] (m2) {\(q\)}; \vertex [above right=of k] (l1) {\(\bar{t}\)}; \vertex [below right=of k] (l2) {\(t\)};
\diagram{ (r1) --[fermion] (p1)--[fermion] (m1); (r2) --[fermion] (p2) --[fermion] (m2); (p1) --[gluon, edge label={\(g\)}] (p) --[gluon, edge label={\(g\)}] (k), (p2) --[gluon, edge label={\(g\)}] (p); (l1) --[fermion] (k) --[fermion] (l2)
}; 
\end{feynman}
\end{tikzpicture}
\begin{tikzpicture}
\begin{feynman}[small]
\vertex (r1) {\(q\)}; 
\vertex [right=of r1] (p1); 
\vertex [below right=of p1] (p11); 
\vertex[below=of p11] (p21); 
\vertex [below left=of p21] (p2); 
\vertex [left=of p2] (r2) {\(q\)}; 
\vertex [right=of p11] (k1) {\(\bar{t}\)}; 
\vertex[right=of p21] (k2) {\(t\)}; 
\vertex[above=of k1] (k11) {\(q\)}; 
\vertex[below=of k2] (k22) {\(q\)};
 \diagram{ (r1) --[fermion] (p1) --[fermion] (k11); 
 (r2) --[fermion] (p2) --[fermion] (k22); (p1)--[gluon, edge label={\(g\)}] (p11); (p2) --[gluon, edge label={\(g\)}] (p21); (k1) --[fermion] (p11) --[fermion, edge label={\(t\)}] (p21) --[fermion] (k2)
 };
\end{feynman}
\end{tikzpicture}
\caption{Top-quark pair production in proton-proton collisions.}
\label{fig:: top pair pp}
\end{figure}
\end{flushleft}
\begin{flushleft}
As mentioned, the top quark can also be produced as a single top quark through the weak interaction. A flavor changing $W$-boson can decay into a $\bar{b}t$ pair, or $b$ can emit a $t$ quark. Possible diagrams in proton-proton and proton-antiproton collisions are shown in Fig. (\ref{fig:: single top}).
\begin{figure}[H]
\begin{tikzpicture}
\begin{feynman}
\vertex (r1) {\(q\)}; \vertex [below right=of r1] (r); \vertex [below left=of r] (r2) {\(\bar{q}\)};\vertex [right=of r] (p); \vertex [above right=of p] (p1) {\(\bar{b}\)}; \vertex [below right=of p] (p2) {\(t\)};
\diagram{ (r1) --[fermion] (r) --[fermion] (r2); (r) --[photon, edge label={\(W\)}] (p); (p1) --[fermion] (p) --[fermion] (p2)
};
\end{feynman}
\end{tikzpicture}
\begin{tikzpicture}
\begin{feynman}
\vertex (r1) {\(g\)}; 
\vertex [below right=of r1] (r); 
\vertex [below left=of r] (r2) {\(b\)};\vertex [right=of r] (p); 
\vertex [above right=of p] (p1) {\(t\)}; 
\vertex [below right=of p] (p2) {\(W\)};
\diagram{ (r2) --[fermion] (r) --[gluon] (r1); (r) --[fermion, edge label={\(b\)}] (p); (p2) --[scalar] (p) --[fermion] (p1)
};
\end{feynman}
\end{tikzpicture}
\caption{Single top-quark production in proton-proton and proton-antiproton collisions.}
\label{fig:: single top}
\end{figure}
\end{flushleft}


\subsubsection*{Decay}

\begin{flushleft}
The top quark is so heavy and shortlived that it does not have time to form hadrons before it decays. The predicted lifetime according to the standard model is $5 \cdot 10^{-25}$ s, which is only about a twentieth of the timescale for strong interactions. The top quark decays through the weak interaction to a $W$-boson and a down-type quark, i.e. down, strange or bottom. One-loop diagrams are very supressed, so they are not included here. The top is identified experimentally by measuring the spin states of the produced $W$ and $b$.
\begin{align*}
t \rightarrow W^+ b,s,d.
\end{align*}

\begin{figure}[H]
\centering
\begin{tikzpicture}
\begin{feynman}
\vertex (a) {\(t\)}; \vertex [right=of a] (b); \vertex [above right= of b] (c); \vertex [below right=of b] (d) {\(d,s,b\)};
\diagram{
(a) --[fermion] (b) --[photon, edge label= {\(W^+\)}] (c), (b) --[fermion] (d)};
\end{feynman}
\end{tikzpicture}
\caption{Top quark decay}
\label{fig::top quark decay}
\end{figure}
We consider a top candidate event, as shown in Fig. (\ref{fig:: top event ATLAS}). The electron is shown by the green track and calorimeter cluster in the 3D view, and the muon by the long red track intersecting the muon chambers. The two b-tagged jets are shown by the purple cones, whose sizes are proportional to the jet energies. The inset shows the XY view of the vertex region, with the secondary vertices of the two b-tagged jets indicated by the orange ellipses.
\begin{figure}[H]
\centering
\includegraphics[scale=0.15]{top_event.png}
\caption{ATLAS top event: top pair e-mu dilepton candidate with two b-tagged jets.}
\label{fig:: top event ATLAS}
\end{figure}
\end{flushleft}



\subsection*{Search for new physics}
\begin{flushleft}
The lightest particles in supersymmetry are the neutralino, $\tilde{\chi}_1^0$, which is assumed to be stable, and the lightest chargino $\tilde{\chi}_1^{\pm}$. We consider the supersymmetric process that includes neutralinos and charginos
\begin{align}\label{susy eq 1}
pp \rightarrow \chi^+_1 \chi^0_2 \rightarrow WZ \chi_1^0 \chi_1^0.
\end{align}
In $pp$-collisions top-squarks can be produced. These can decay into a top or bottom quark and a neutralino (sett inn her), namely
\begin{align*}
\tilde{t}_1 \rightarrow t,b + \tilde{\chi}_1^0.
\end{align*}
As can be seen in Fig. (\ref{fig::top quark decay}) a top and antibottom can become a $W$-boson, so that the second transition in process (\ref{susy eq 1}) can come from two of these $W$ bosons fusioning into a $Z$-boson. Previous ATLAS analyses using data at $\sqrt{s}=7$ TeV (center of mass energy) and 8 TeV have places exclusions limits at $95\%$ confidence level (CL) on both the $\tilde{t}_1 \rightarrow b + \tilde{\chi}_1^{\pm}$ and $\tilde{t}_1 \rightarrow t+ \tilde{\chi}_1^0$ decay modes [cite Atlas].
\end{flushleft}

\section{Gauge theories}

\subsection*{Standard model}
\begin{flushleft}
The particles in the Standard model are organized in a threefold family structure
\begin{align*}
\begin{pmatrix}
\nu_e & u\\
e^- & d'\\
\end{pmatrix}, 
\begin{pmatrix}
\nu_{\mu} & c\\
\mu^- & s'\\
\end{pmatrix},
\begin{pmatrix}
\nu_{\tau} & t\\
\tau^- & b'\\
\end{pmatrix}.
\end{align*}
\end{flushleft}

\subsection*{Grand Unifying Theories}
\begin{flushleft}
There is a grand unifying theory, that suggests that the coupling constants converge at a given energy scale. The proposed scale is at around $10^{15}$ GeV. The simplest GUT is the $U(5)$ symmetry, which contains the standard model $SU(3) \otimes SU(2) \otimes U(1)$. Group symmetries such as this allow for known particles to be interpreted as different states of a single particle field.
\end{flushleft}

\subsection*{Gauge and symmetries}

\begin{flushleft}
The gauge principle is the requirement that a Lagrangian, e.g. the free Dirac fermion Lagrangian
\begin{align*}
\mathcal{L}_0 = \bar{\psi}(i \partial_{\mu} - m) \psi,
\end{align*}
which is symmetric under $U(1)$ \textit{global} transformations also be \textit{symmetric under local transformations}.
\end{flushleft}

\begin{flushleft}
The standard model is based on a non-Abelian group of transformations, called the Lorentz group, or $SU(3) \otimes U(1)$. The spinors are symmetric under local transformations
\begin{align*}
\psi \rightarrow \psi'=e^{iQ\theta} \Psi,
\end{align*}
and the theory is gauged by promoting these symmetries to local symmetries, i.e.
\begin{align*}
\theta \rightarrow \theta (x).
\end{align*}
In order to preserve symmetries we must introduce the covariant derivative
\begin{align}
D_{\mu}(x) = \partial_{\mu} - ieQA_{\mu}(x)
\end{align}
\end{flushleft}

\begin{flushleft}
Symmetries are related to conserved quantities through Noether's theorem, which claims that for every symmetry there must be a conserved quantity.
\end{flushleft}



\subsection{QCD}
\begin{flushleft}
From experiments we know that hadronic matter is made up of quarks -- mesons are $q \bar{q}$ and baryons are $qqq$. Because of Fermi statistics we need another quantum number, which we call colour. There are three colors
\begin{align*}
\alpha, \beta, \gamma = \text{ red, green, blue},
\end{align*}
and all asymptotic states are colorless. We take color to be the quantum number of strong interactions.
\end{flushleft}

\subsubsection*{QCD Lagrangian}

\begin{flushleft}
Consider the free Lagrangian for quarks of color $\alpha$ and flavor $f$
\begin{align*}
\mathcal{L}_0 = \sum_f \bar{q}_f^{\alpha} (i \cancel{\partial} - m_f)q_f^{\alpha}.
\end{align*}
This Lagrangian is symmetric under global $SU(3)_C$ transformations in color space, and we wish to promote these to local symmetries. The derivative is then no longer invariant, so we must derive a \textit{covariant derivative} which transforms in the same way as $q_f^{\alpha}$ under local transformations. The transformations are given by
\begin{align*}
U = \exp \Big( i \frac{\lambda^a}{2} \theta_a \Big),
\end{align*}
where $\lambda^a$ are the 8 generators of the group. Since this is a non-Abelian group the generators don't commute
\begin{align*}
\Big[ \frac{\lambda^a}{2}, \frac{\lambda^b}{2}
\Big] = i f^{abc} \frac{\lambda^c}{2},
\end{align*}
where $f^{abc}$ are the structure constants. The covariant derivative is
\begin{align*}
D^{\mu} q_f = \Big[ \partial^{\mu} + ig_s \frac{\lambda^a}{2} G^{\mu}_a (x) \Big] qf \equiv [\partial^{\mu} + ig_s G^{\mu} (x)] q_f,
\end{align*}
where $G^{\mu}_a$ are the eight gluons, or gauge bosons, corresponding to the eight generators. Because we want $D^{\mu}$ to transform exactly like the quarks, the transformation of the gluons is fixed
\begin{align*}
D^{\mu} & \rightarrow D^{\mu '} = U D^{\mu} U^{\dagger}\\
G^{\mu} & \rightarrow G^{\mu '} = U G^{\mu} U^{\dagger} + \frac{i}{g_s} (\partial^{\mu} U) U^{\dagger}.
\end{align*}
For infinitesimal $SU(3)_C$ transformations we therefore get
\begin{align*}
q_f^{\alpha} &\rightarrow q_f^{\alpha '} = q_f^{\alpha} + i \Big( \frac{\lambda^a}{2} \Big)_{\alpha \beta} \delta \theta_a q_f^{\beta}\\
G_a^{\mu} & \rightarrow G_a^{\mu '} = G_a^{\mu} - \frac{1}{g_s} \partial^{\mu} (\delta \theta_a) - f^{abc} \delta \theta_b G_c^{\mu}.
\end{align*}
\end{flushleft}

\begin{flushleft}
In order to construct a Lagrangian we must see which terms are allowed. Only terms symmetric under local $SU(3)_c$ transformations are allowed, to there is no mass term e.g.
\begin{align*}
G_{\mu} G^{\mu} &\rightarrow (G_{\mu})'(G^{\mu})' = (U G_{\mu} U^{\dagger} + \frac{i}{g_s} (\partial_{\mu} U) U^{\dagger})\\
& \times (U G^{\mu} U^{\dagger} + \frac{i}{g_s} (\partial^{\mu} U) U^{\dagger})\\
&= U G_{\mu} U^{\dagger} U G^{\mu} U^{\dagger} + \frac{i}{g_s}U G_{\mu} U^{\dagger}  (\partial^{\mu} U)U^{\dagger}\\
&+  \frac{i}{g_s} (\partial_{\mu} U) U^{\dagger}U G^{\mu} U^{\dagger}\\
& - \frac{1}{g_s^2} (\partial_{\mu} U) U^{\dagger} (\partial^{\mu} U) U^{\dagger}\\
&= U G_{\mu}  G^{\mu} U^{\dagger} + \frac{i}{g_s}U G_{\mu} U^{\dagger}  (\partial^{\mu} U)U^{\dagger}\\
&+  \frac{i}{g_s} (\partial_{\mu} U) G^{\mu} U^{\dagger} - \frac{1}{g_s^2} (\partial_{\mu} U) U^{\dagger} (\partial^{\mu} U) U^{\dagger} \neq G_{\mu}G^{\mu}\\
\end{align*}
since $U^{\dagger} U = U^{\dagger} U = 1$. So no mass term is allowed, and the gluons are massless.
\end{flushleft}

\pagebreak

\begin{flushleft}
 We instead consider kinetic terms, analogous to QED
 \end{flushleft}
\end{multicols}
\begin{align*}
i g_s G_{\mu \nu} &= [D_{\mu}, D_{\nu}] = (\partial_{\mu} + ig_s G_{\mu})(\partial_{\nu} + ig_s G_{\nu} )- (\partial_{\nu} + ig_s G_{\nu} ) (\partial_{\mu} + ig_s G_{\mu} )\\
&= \partial_{\mu} \partial_{\nu}+ ig_s \partial_{\mu}  G_{\nu} +ig_s G_{\mu} \partial_{\nu}  - g_s^2 G_{\mu} G_{\nu}
- \partial_{\nu} \partial_{\mu}- ig_s\partial_{\nu}  G_{\mu} -ig_s G_{\nu} \partial_{\mu}  + g_s^2 G_{\nu} G_{\mu}\\
&= [\partial_{\mu}, \partial_{\nu}]+ ig_s (\partial_{\mu}  G_{\nu} - \partial_{\nu}  G_{\mu}) +ig_s (G_{\mu} \partial_{\nu}- G_{\nu} \partial_{\mu})  + g_s^2 [G_{\mu}, G_{\nu}]\\
&= ig_s (\partial_{\mu}  G_{\nu} - \partial_{\nu}  G_{\mu})   + g_s^2 [G_{\mu}, G_{\nu}]\\
& \rightarrow \frac{\lambda^a}{2} G_a^{\mu \nu} \equiv \partial_{\mu} G_{\nu} - \partial_{\nu} G_{\mu} + g_s [G_{\mu}, G_{\nu}],\\
G_a^{\mu \nu} (x) &=  \partial^{\mu} G^{\nu}_a - \partial^{\nu} G^{\mu}_a - g_s f^{abc} G^{\mu}_b G^{\nu}_c
\end{align*}
\begin{multicols}{2}
\begin{flushleft}
where we've used that
\begin{align*}
[\partial_{\mu}, \partial_{\nu}] &= 0, \text{ } (G_{\mu} \partial_{\nu}- G_{\nu} \partial_{\mu})=0,\\
\end{align*}
because partial derivatives commute. Under $SU(3)_C$ this term transforms in the desired way
\begin{align*}
G^{\mu \nu} \rightarrow G^{\mu \nu '} = U G^{\mu \nu} U^{\dagger}.
\end{align*}
\end{flushleft}

\begin{flushleft}
We now have the terms for the QCD Lagrangian 
\begin{align}
\mathcal{L}_{QCD} \equiv - \frac{1}{4} G^{\mu \nu}_a G^a_{\mu \nu} + \sum_f \bar{q}_f (i \gamma^{\mu} D_{\mu} -m_f)q_f.
\end{align}
We now note a difference between QED and QCD. In QED the gauge bosons commute, and so the field strength only contains terms of te first order in $A_{\mu}$. In QCD, however, the field strength contains a term $G^{\mu}_b G^{\nu}_c$, which means that the Lagrangian has terms that have third and fourth order terms in the gauge boson - meaning that we have self-interaction vertices with three and four gluons, as seen in figure (\ref{QCD vertices}).
 
\begin{figure}[H]
\begin{tikzpicture}
\begin{feynman}
\vertex (a); \vertex [below right=of a] (r); \vertex [above right=of r] (b); \vertex [below left=of r] (c); \vertex [below right=of r] (d); \vertex [left=of r] (D);
\diagram{ (a) --[gluon] (r) -- [gluon] (b); (c)--[gluon] (r) --[gluon] (d)
};
\end{feynman}
\end{tikzpicture}
\begin{tikzpicture}
\begin{feynman}
\vertex (a); \vertex [below right=of a] (r); \vertex [above right=of r] (b); \vertex [below=of r] (c); 
\diagram{ (a) --[gluon] (r) -- [gluon] (b); (c)--[gluon] (r)
};
\end{feynman}
\end{tikzpicture}
\begin{tikzpicture}
\begin{feynman}
\vertex (a); \vertex [below right=of a] (r); \vertex [above right=of r] (b); \vertex [below=of r] (c); 
\diagram{ (a) --[gluon] (r) -- [gluon] (b); (c)--[fermion] (r)
};
\end{feynman}
\end{tikzpicture}
\caption{The interaction vertices from the QCD Lagrangian.}
\label{QCD vertices}
\end{figure}
\end{flushleft}

\subsection*{QCD and QED}
\begin{flushleft}
Conceptually, the main difference between QCD and QED are the symmetries they are based on. QCD is a non-Abelian gauge theory, while QED is an Abelian theory. As mentioned in the derivation of $\mathcal{L}_{QCD}$ this means that the generators of $QCD$ don't commute, while the ones in $QED$ do. This leads to the presence of self-interaction vertices in QCD, which could explain asymptotic freedom (antiscreening makes strong interactions weaker at short distances) and confinement (we can't observe free quarks, because strong forces become strong at long distances).
\end{flushleft}

\begin{flushleft}
Experimentally, the strong force is perhaps more difficult to detect because of confinement. As quarks are brought apart, the force between them becomes stronger. This is why we don't observe free quarks. In QED, however, we also have particles such as electrons and muons, which \textit{are} observed freely.
\end{flushleft}


\subsection*{Electroweak unification}
\begin{flushleft}
We consider a symmetry group with doublets, $SU(2)$, because we wish to distinguish between left- and righthanded particles. We also want to include electromagnetism, so the gauge group we use is
\begin{align}
G \equiv SU(2)_L \otimes U(1)_Y,
\end{align}
where $L$ refers to lefthanded fields and $Y$ is the hypercharge. To describe the quark and lepton families we use the following notation
\begin{align}
\text{Quark: }\psi_1(x)&= \begin{pmatrix}
u\\
d\\
\end{pmatrix}_L, 
\psi_2(x)=u_R, \psi_3 (x) = d_R,\\
\text{Lepton: } \psi_1(x) &= \begin{pmatrix}
\nu_e\\
e^-\\
\end{pmatrix}_L, \psi_2 (x) = \nu_{eR}, \psi_3(x)=e^-_R.
\end{align}
The free Lagrangian is
\begin{align*}
\mathcal{L}_0 &= i \bar{u}(x) \gamma^{\mu} \partial_{\mu} u(x) + i \bar{d}(x) \gamma^{\mu} \partial_{\mu} d(x)= \sum_{j=1}^3 i \bar{\psi}_j(x) \gamma^{\mu} \partial_{\mu} \psi_j(x).
\end{align*}
This Lagrangian is invariant under global $G$ transformations i flavour space
\begin{align*}
\psi_1(x) &\rightarrow \psi_1' \equiv \exp\{iy_1 \beta\} U_L \psi_1(x),\\
\psi_2(x) &\rightarrow \psi_2' \equiv \exp\{iy_2 \beta\} \psi_2(x),\\
\psi_3(x) &\rightarrow \psi_3' \equiv \exp\{iy_3 \beta\}\psi_3(x).
\end{align*}
The generators of the gauge transformation $U_L$ are the Pauli matrices, and $U_L$ only acts on the doublet field $\psi_1$. Including a mass term would mix left- and righthanded states and spoil symmetry, and is therefore left for later consideration. We promote the symmetries to local symmetries, with the four gauge parameters $\alpha^i(x)$ and $\beta (x)$
\begin{align*}
D_{\mu} \psi_1(x) & \equiv \Big[\partial_{\mu} + ig \tilde{W}_{\mu}(x) + ig' y_1 B_{\mu}(x) \Big] \psi_1(x),\\
D_{\mu} \psi_3(x) & \equiv [\partial_{\mu} + ig'y_2 B_{\mu}(x)] \psi_2(x),\\
D_{\mu} \psi_3(x) & \equiv [\partial_{\mu} + ig'y_3 B_{\mu}(x)] \psi_3(x),
\end{align*}
where
\begin{align*}
\tilde{W}_{\mu}(x) \equiv \frac{\sigma_i}{2} W^i_{\mu}(x).
\end{align*}
The transformation properties of the fields are fixed by the invariance of the covariant derivatives
\begin{align*}
B_{\mu} (x) & \rightarrow B_{\mu}' \equiv B_{\mu}(x)-\frac{1}{g'} \partial_{\mu} \beta (x),\\
\tilde{W}_{\mu} & \rightarrow \tilde{W}_{\mu}' \equiv U_L(x) \tilde{W}_{\mu} U_L^{\dagger} (x) +\frac{i}{g} \partial_{\mu} U_L(x) U_L^{\dagger} (x).
\end{align*}
$B_{\mu}$ transforms identically to the QED photon, and $W_{\mu}^i$ transform analogously to the gluon fields in QCD. We now find the kinetic terms of the gauge fields by calculating the field strengths
\end{flushleft}
\end{multicols}
\begin{align*}
B_{\mu \nu} & \equiv \partial_{\mu} B_{\nu} - \partial_{\nu} B_{\mu},\\
\tilde{W}_{\mu \nu} & \equiv - \frac{i}{g} \Big[ (\partial_{\mu} + ig \tilde{W}_{\mu}), (\partial_{\nu} + ig \tilde{W}_{\nu}) \Big]\\
&= -\frac{i}{g} \Big((\partial_{\mu} + ig \tilde{W}_{\mu}) (\partial_{\nu} + ig \tilde{W}_{\nu}) - (\partial_{\nu} + ig \tilde{W}_{\nu})(\partial_{\mu} + ig \tilde{W}_{\mu}) \Big)\\
&= -\frac{i}{g} \Big(
\partial_{\mu} \partial_{\nu} + \partial_{\mu}ig \tilde{W}_{\nu}
+ ig \tilde{W}_{\mu} \partial_{\nu} + ig \tilde{W}_{\mu} ig \tilde{W}_{\nu}
- \partial_{\nu}\partial_{\mu} - \partial_{\nu}ig \tilde{W}_{\mu}
- ig \tilde{W}_{\nu}\partial_{\mu} - ig \tilde{W}_{\nu} ig \tilde{W}_{\mu} 
\Big)\\
&= -\frac{i}{g} \Big(
 ig(\partial_{\mu} \tilde{W}_{\nu}
 - \partial_{\nu} \tilde{W}_{\mu})
+ ig (\tilde{W}_{\mu} \partial_{\nu}
- \tilde{W}_{\nu}\partial_{\mu} )
+ [\partial_{\mu}, \partial_{\nu}]
+ g^2 [\tilde{W}_{\nu}, \tilde{W}_{\mu}] 
\Big)\\
&= \partial_{\mu} \tilde{W}_{\nu}
 - \partial_{\nu} \tilde{W}_{\mu}
+i g [\tilde{W}_{\mu}, \tilde{W}_{\nu}]. \\
\end{align*}

\begin{multicols}{2}
\begin{flushleft}
Since $SU(2)_L$ is non-Abelian, this last commutation relation is not zero. In component form the kinetic term is given by
\begin{align*}
\tilde{W}_{\mu \nu} \equiv \frac{\sigma_i}{2} W_{\mu \nu}^i; \text{ } W^i_{\mu \nu} = \partial_{\mu} W_{\nu}^i- \partial_{\nu} W_{\mu}^i - g \epsilon^{ijk}W_{\mu}^jW_{\nu}^k.
\end{align*}
$\tilde{W}_{\mu \nu}$ transforms covariantly under gauge transformations, so we take the trace, and get the full kinetic Lagrangian
\begin{align}\label{EW kin}
\mathcal{L}_{Kin} &= - \frac{1}{4} B_{\mu \nu} B^{\mu \nu} - \frac{1}{4} W_{\mu \nu}^i W^{\mu \nu}_i.
\end{align}
Since $W_{\mu\nu}$ contains quadratic terms in the gauge fields, as in QCD, we get cubic and quartic terms. 
\end{flushleft}

\subsubsection{Currents and self-interactions}
\begin{flushleft}
A guage invariant lagrangian is
\begin{align*}
\mathcal{L} = \sum_{j=1}^3 i \bar{\psi}_j(x) \gamma^{\mu} D_{\mu} \psi_j(x).
\end{align*}
From here we get the interactions of fermion fields with gauge bosons
\begin{align*}
\tilde{W}_{\mu} = \frac{\sigma^i}{2} W_{\mu}^i = \frac{1}{2} \begin{pmatrix}
W_{\mu}^3 & \sqrt{2} W_{\mu}^{\dagger}\\
\sqrt{2}W_{\mu} & -W_{\mu}^3\\
\end{pmatrix},
\end{align*}
where $W_{\mu}^{\dagger} \equiv (W_{\mu}^1-iW_{\mu}^2)/\sqrt{2}$. The couplings are shown in Fig. (\ref{fig::EW fermion-gauge}).
\end{flushleft}
\begin{figure}[H]
\centering
\begin{tikzpicture}
\begin{feynman}
\vertex (a); \vertex [above right=of a] (b); \vertex [below right=of b] (c); \vertex [above=of b] (d);
\diagram{
(a) --[fermion, edge label={\(q_d, l^-\)}] (b) --[fermion, edge label={\(q_u, \nu_l^-\)}] (c); (b) --[photon, edge label={\(W\)}] (d)
};
\end{feynman}
\end{tikzpicture}
\caption{Electroweak vertex between fermions and charged gauge bosons.}
\label{fig::EW fermion-gauge}
\end{figure}

\begin{flushleft}
The neutral currents are mediated by the photon and $Z$ boson, which we now wish to identify with the bosons in the Lagrangian $W_{\mu}^3$ and $B_{\mu}$. Since both of these fields are neutral we try a random combination of them, expressed in terms of the angle $\theta_W$
\begin{align*}
\begin{pmatrix}
W_{\mu}^3\\
B_{\mu}
\end{pmatrix}
\equiv \begin{pmatrix}
\cos \theta_W & \sin \theta_W\\
- \sin \theta_W & \cos \theta_W
\end{pmatrix}
\begin{pmatrix}
Z_{\mu}\\
A_{\mu}
\end{pmatrix},
\end{align*}
where the Z boson aquires mass through spontaneous symmetry breaking. We get the neutral current Lagrangian (drop the explicit x-dependence from here)
\begin{align*}
\mathcal{L} &= \sum_{j=1}^3 i \bar{\psi}_j  \gamma^{\mu} D_{\mu} \psi_j\\
&= - g\bar{\psi}_1 \gamma^{\mu} \tilde{W}_{\mu} \psi_1(x) 
+ \sum_{j=1}^3 i\bar{\psi}_j \gamma^{\mu}(\partial_{\mu} + ig' y_jB_{\mu}) \psi_j\\
\mathcal{L} & \supset - g\bar{\psi}_1 \gamma^{\mu} \begin{pmatrix}
W_{\mu}^3 \psi_1^u + \sqrt{2} W_{\mu}^{\dagger} \psi_1^d\\
\sqrt{2} W_{\mu} \psi_1^u - W_{\mu}^3 \psi_1^d\\
\end{pmatrix}  
-g' B_{\mu} \sum_{j=1}^3 i\bar{\psi}_j \gamma^{\mu} y_j \psi_j\\
\mathcal{L} & \supset  - g\bar{\psi}_1 \gamma^{\mu} 
W_{\mu}^3 
\begin{pmatrix}
\psi_1^u \\
- \psi_1^d\\
\end{pmatrix}
-g' B_{\mu} \sum_{j=1}^3 i\bar{\psi}_j \gamma^{\mu} y_j \psi_j\\
&= - g W_{\mu}^3 \bar{\psi}_1 \gamma^{\mu} 
 \sigma_3 \psi_1
-g' B_{\mu} \sum_{j=1}^3 i\bar{\psi}_j \gamma^{\mu} y_j \psi_j\\
&= - g (\cos \theta_W Z_{\mu} + \sin \theta_W A_{\mu}) \bar{\psi}_1 \gamma^{\mu} 
 \sigma_3 \psi_1\\
& -g' (- \sin \theta_W Z_{\mu} + \cos \theta_W A_{\mu}) \sum_{j=1}^3 i\bar{\psi}_j \gamma^{\mu} y_j \psi_j\\
&= -\bar{\psi}_1 \gamma^{\mu}  [ Z_{\mu}(\sigma_3 g \cos \theta_W - g' y_1 \sin \theta_W)\\&+ A_{\mu} (\sigma_3 g \sin \theta_W + g' y_1 (\cos \theta_W)] 
 \psi_1\\
& -g' (- \sin \theta_W Z_{\mu} + \cos \theta_W A_{\mu}) \sum_{j=2}^3 i\bar{\psi}_j \gamma^{\mu} y_j \psi_j.\\
\end{align*}
The feynman diagrams of these vertices can be found in Fig. (\ref{fig:: EW neutral}).
\begin{figure}[H]
\centering
\begin{tikzpicture}
\begin{feynman}
\vertex (a); \vertex [above right=of a] (b); \vertex [below right=of b] (c); \vertex [above=of b] (d); 
\diagram{ (a) --[fermion, edge label={\(f\)}] (b) --[fermion, edge label={\(f\)}] (c), (b) --[photon, edge label={\(\gamma, Z\)}] (d)
};
\end{feynman}
\end{tikzpicture}
\caption{Electroweak vertex between fermions and neutral bosons.}
\label{fig:: EW neutral}
\end{figure}
\end{flushleft}

\begin{flushleft}
The electroweak Lagrangian also generates cubic and quartic terms, from the non-zero commutator in $\frac{1}{2} tr[\tilde{W}_{\mu \nu} \tilde{W}^{\mu \nu}]$
\end{flushleft}
\end{multicols}

\begin{align*}
tr[\tilde{W}_{\mu \nu} \tilde{W}^{\mu \nu}] &= (\partial_{\mu} \tilde{W}_{\nu}^i 
- \partial_{\nu} \tilde{W}_{\mu}^i 
- g \epsilon^{ijk} W_{\mu}^j W_{\nu}^k )
(\partial_{\mu} \tilde{W}^{\nu}_i 
- \partial^{\nu} \tilde{W}^{\mu}_i 
- g \epsilon^{ijk} W^{\mu}_j W^{\nu}_k) \\
& \supset (\partial_{\mu} \tilde{W}_{\nu}^i 
- \partial_{\nu} \tilde{W}_{\mu}^i)(- g \epsilon^{ijk} W^{\mu}_j W^{\nu}_k)
- g \epsilon^{ijk} W_{\mu}^j W_{\nu}^k  ()
\end{align*} 

\begin{flushleft}
From this epxression we can get the cubic and quartic terms, some of the calculations are done in Appendix A
\begin{align*}
\mathcal{L}_3 
&= ie cot \theta_W \Big( (\partial^{\mu} W^{\nu}- \partial^{\nu} W^{\mu})W_{\mu}^{\dagger} Z_{\nu} 
- \big( \partial^{\mu} W^{\nu \dagger} - \partial^{\nu} W^{\mu \dagger} \big) W_{\mu} Z_{\nu} 
+ W_{\mu} W_{\nu}^{\dagger} (\partial^{\mu} Z^{\nu} - \partial^{\nu} Z^{\mu}) \Big)\\
&+ ie \Big( (\partial^{\mu} W^{\nu} - \partial^{\nu} W^{\mu}) W_{\mu}^{\dagger} A_{\nu} 
- \Big( \partial^{\mu} W^{\nu \dagger} - \partial^{\nu} W^{\mu \dagger} \Big)W_{\mu} A_{\nu} 
+ W_{\mu} W_{\nu}^{\dagger} (\partial^{\mu} A^{\nu}- \partial^{\nu} A^{\mu}) \Big)\\
\mathcal{L}_4 &= - \frac{e^2}{2 \sin^2 \theta_W} \Big( (W_{\mu}^{\dagger}W^{\mu})^2 - W_{\mu}^{\dagger} W^{\mu} W_{\nu} W^{\nu} \Big) - e^2 \cot^2 \theta_W \Big( W_{\mu}^{\dagger}W^{\mu} Z_{\nu}Z^{\nu} - W_{\mu}^{\dagger} Z^{\mu}W_{\nu} Z^{\nu}  \Big)\\
&+ -e^2 \cot \theta_W \Big( 2 W_{\mu}^{\dagger} W^{\mu} Z_{\nu} A^{\nu} - W_{\mu}^{\dagger} Z^{\mu} W_{\nu} A^{\nu} - W_{mu}^{\dagger}A^{\mu} W_{\nu} Z^{\nu} \Big) - e^2 \Big( 
W-{\mu}^{\dagger} W^{\mu} A_{\nu} A^{\nu}- W_{\mu}^{\dagger}A^{\mu} W_{\nu} A^{\nu} 
\Big)
\end{align*}

\end{flushleft}
\begin{multicols}{2}

\begin{flushleft}

The feynman diagrams for these vertices can be seen in Fig. (\ref{fig:: cubic and quartic in EW}).
\begin{figure}[H]
\centering
\begin{tikzpicture}
\begin{feynman}
\vertex (a); \vertex [right=of a] (b); \vertex [above right=of b] (c); \vertex [below right=of b] (d);
\diagram{
(a) --[photon, edge label={\(\gamma, Z\)}] (b), (c) --[photon, edge label={\(W^+\)}] (b), (d)  --[photon, edge label={\(W^-\)}] (b)
};
\end{feynman}
\end{tikzpicture}
\begin{tikzpicture}
\begin{feynman}
\vertex (a); \vertex [below right=of a] (s); \vertex [above right=of s] (b); 
\vertex [below left=of s] (c); \vertex [below right=of s] (d);
\diagram{
(a) --[photon, edge label={\(W^+\)}] (s), (b) --[photon, edge label={\(W^+, \gamma, Z\)}] (s), (c) --[photon, edge label={\(W^-\)}] (s), (d) --[photon, edge label={\(W^-, \gamma, Z\)}] (s)
};
\end{feynman}
\end{tikzpicture}
\caption{Cubic and quartic vertices in electroweak theory.}
\label{fig:: cubic and quartic in EW}
\end{figure}
\end{flushleft}




\end{multicols}


\cite{Aad2014}

\pagebreak

\section{Sources}

\begin{flushleft}
\textbf{ADD TO BIBTEX!!!!!!!!!!!!!!}
http://download.springer.com/static/pdf/404/art%253A10.1007%252FJHEP06%25282014%2529124.pdf?originUrl=http%3A%2F%2Flink.springer.com%2Farticle%2F10.1007%2FJHEP06%282014%29124&token2=exp=1486590441~acl=%2Fstatic%2Fpdf%2F404%2Fart%25253A10.1007%25252FJHEP06%2525282014%252529124.pdf%3ForiginUrl%3Dhttp%253A%252F%252Flink.springer.com%252Farticle%252F10.1007%252FJHEP06%25282014%2529124*~hmac=3096ec23a4e3bbb4210c8ee50145b64953a52f52379657c7a1159a12c818f9e7
\end{flushleft}

\cite{Aad2014}



\section*{Appendix A}

\begin{align*}
tr[\tilde{W}_{\mu \nu} \tilde{W}^{\mu \nu}] &= (\partial_{\mu} \tilde{W}_{\nu}^i 
- \partial_{\nu} \tilde{W}_{\mu}^i 
- g \epsilon^{ijk} W_{\mu}^j W_{\nu}^k )
(\partial^{\mu} \tilde{W}^{\nu}_i 
- \partial^{\nu} \tilde{W}^{\mu}_i 
- g \epsilon_{ilm} W^{\mu}_l W^{\nu}_m) \\
& \supset (\partial_{\mu} W_{\nu}^i 
- \partial_{\nu} W_{\mu}^i)(- g \epsilon^{ijk} W^{\mu}_j W^{\nu}_k)
- g \epsilon^{ijk} W_{\mu}^j W_{\nu}^k  (\partial^{\mu} W^{\nu}_i 
- \partial^{\nu} W^{\mu}_i )
- g \epsilon^{ijk} W_{\mu}^j W_{\nu}^k (- g \epsilon_{ilm} W^{\mu}_l W^{\nu}_m))\\
&= - g \epsilon^{ijk} W^{\mu}_j W^{\nu}_k(\partial_{\mu} W_{\nu}^i 
- \partial_{\nu} W_{\mu}^i)
- g \epsilon_{ijk} W_{\mu}^j W_{\nu}^k  (\partial^{\mu} W^{\nu}_i 
- \partial^{\nu} W^{\mu}_i )
+ g^2 \epsilon^{ijk}  \epsilon_{ilm} W_{\mu}^j W_{\nu}^k W^{\mu}_l W^{\nu}_m
\end{align*} 
Indices that are summed over are dummy indices, and can be changed
\begin{align*}
&= - g \epsilon^{ijk} W^{\mu}_j W^{\nu}_k(\partial_{\mu} W_{\nu}^i 
- \partial_{\nu} W_{\mu}^i)
- g \epsilon^{ijk} W^{\mu}_j W^{\nu}_k  (\partial_{\mu} W_{\nu}^i 
- \partial_{\nu} W_{\mu}^i )
+ g^2 [\delta_j^l \delta_k^m - \delta_j^m \delta_k^l] W_{\mu}^j W_{\nu}^k W^{\mu}_l W^{\nu}_m\\
&= - 2 g \epsilon^{ijk} W^{\mu}_j W^{\nu}_k(\partial_{\mu} W_{\nu}^i 
- \partial_{\nu} W_{\mu}^i)
+ g^2 (W_{\mu}^j W^{\mu}_j W_{\nu}^k W^{\nu}_k - W_{\mu}^j W^{\nu}_j W_{\nu}^k W^{\mu}_k )\\
\end{align*}
Write out the first term
\begin{align*}
& \epsilon^{ijk} W^{\mu}_j W^{\nu}_k(\partial_{\mu} W_{\nu}^i 
- \partial_{\nu} W_{\mu}^i)\\
&= W^{\mu}_2 W^{\nu}_3(\partial_{\mu} W_{\nu}^1 
- \partial_{\nu} W_{\mu}^1)
- W^{\mu}_3 W^{\nu}_2(\partial_{\mu} W_{\nu}^1 
- \partial_{\nu} W_{\mu}^1)
+ W^{\mu}_3 W^{\nu}_1(\partial_{\mu} W_{\nu}^2 
- \partial_{\nu} W_{\mu}^2)\\
&- W^{\mu}_1 W^{\nu}_3(\partial_{\mu} W_{\nu}^2 
- \partial_{\nu} W_{\mu}^2)
+ W^{\mu}_1 W^{\nu}_2(\partial_{\mu} W_{\nu}^3 
- \partial_{\nu} W_{\mu}^3)
- W^{\mu}_2 W^{\nu}_1(\partial_{\mu} W_{\nu}^3 
- \partial_{\nu} W_{\mu}^3)\\
&= W^{\nu}_3 \big( W^{\mu}_2 (\partial_{\mu} W_{\nu}^1 
- \partial_{\nu} W_{\mu}^1)
- W^{\mu}_1 (\partial_{\mu} W_{\nu}^2 
- \partial_{\nu} W_{\mu}^2)
-  W^{\mu}_2(\partial_{\nu} W_{\mu}^1 
- \partial_{\mu} W_{\nu}^1)
+ W^{\mu}_1(\partial_{\nu} W_{\mu}^2 
- \partial_{\nu} W_{\mu}^2) \big)\\
&+ W^{\mu}_1 W^{\nu}_2(\partial_{\mu} W_{\nu}^3 
- \partial_{\nu} W_{\mu}^3)
+ W^{\mu}_1 W^{\nu}_2 (\partial_{\mu} W_{\nu}^3- \partial_{\nu} W_{\mu}^3 )\\
&= 2 \Big[ W^{\nu}_3 \big( W^{\mu}_2 (\partial_{\mu} W_{\nu}^1 
- \partial_{\nu} W_{\mu}^1)
+ W^{\mu}_1 (\partial_{\nu} W_{\mu}^2 -\partial_{\mu} W_{\nu}^2 ) \big)
+ W^{\mu}_1 W^{\nu}_2(\partial_{\mu} W_{\nu}^3 
- \partial_{\nu} W_{\mu}^3) \Big]\\
\end{align*}



NOTE I:
\begin{align*}
(\partial^{\mu} W^{\nu} - \partial^{\nu} W^{\mu}) W^{\dagger}_{\mu} 
&= \big( \partial^{\mu} ((W^{1 \nu}+i W^{2 \nu})/\sqrt{2}) - \partial^{\nu} ((W^{1 \mu}+i W^{2 \mu})/\sqrt{2})\big) (W^1_{\mu}-i W^2_{\mu})/ \sqrt{2}\\
&= \frac{1}{2} \big( \partial^{\mu} W^{1 \nu}+i  \partial^{\mu}W^{2 \nu} - \partial^{\nu}W^{1 \mu}-i \partial^{\nu} W^{2 \mu} \big) (W^1_{\mu}-i W^2_{\mu})\\
(\partial^{\mu}W^{\nu \dagger} - \partial^{\nu} W^{\mu \dagger}) W_{\mu} 
&= \frac{1}{2} \big(
\partial^{\mu}(W^{1 \nu}-i W^{2 \nu}) - \partial^{\nu} (W^{1 \mu}-i W^{2 \mu})
 \big) (W_{\mu}^1+i W_{\mu}^2) \\
 &= \frac{1}{2} \big(
\partial^{\mu}W^{1 \nu}-i \partial^{\mu}W^{2 \nu} - \partial^{\nu} W^{1 \mu} + i \partial^{\nu}W^{2 \mu}
 \big) (W_{\mu}^1+i W_{\mu}^2) \\
\rightarrow  (\partial^{\mu} W^{\nu} - \partial^{\nu} W^{\mu}) W^{\dagger}_{\mu} - (\partial^{\mu}W^{\nu \dagger} - \partial^{\nu} W^{\mu \dagger}) W_{\mu}
&= \frac{1}{2} \big( \partial^{\mu} W^{1 \nu}+i  \partial^{\mu}W^{2 \nu} - \partial^{\nu}W^{1 \mu}-i \partial^{\nu} W^{2 \mu} \big) (W^1_{\mu}-i W^2_{\mu}) \\
&- \frac{1}{2} \big(
\partial^{\mu}W^{1 \nu}-i \partial^{\mu}W^{2 \nu} - \partial^{\nu} W^{1 \mu} + i \partial^{\nu}W^{2 \mu}
 \big) (W_{\mu}^1+i W_{\mu}^2)\\
 &= \frac{1}{2} \big(2i \partial^{\mu}W^{2 \nu}  - 2i \partial^{\nu}W^{2 \mu} \big) W_{\mu}^1 
 - \big( 
2 \partial^{\mu}W^{1 \nu} + 2\partial^{\nu} W^{1 \mu}
\big)i W_{\mu}^2\\
&= i \Big[ (\partial^{\mu}W^{2 \nu}  - \partial^{\nu}W^{2 \mu} ) W_{\mu}^1 
 - ( \partial^{\mu}W^{1 \nu} + \partial^{\nu} W^{1 \mu}
)W_{\mu}^2 \Big]\\
\end{align*}



\subsection{Levi civita}
\begin{flushleft}
The Levi-Civita symbol in three dimensions has the following properties 
\begin{align*}
\epsilon_{ijk} \epsilon^{imn}&= \delta_j^m \delta_k^n - \delta_j^n \delta_k^m\\
\epsilon_{jmn} \epsilon^{imn} &= 2 \delta_j^i\\
\epsilon_{ijk} \epsilon^{ijk} &= 6.
\end{align*}
\end{flushleft}

\subsection{Definitions}
\begin{flushleft}
\begin{align*}
W_{\mu} &= (W^1_{\mu}+i W^2_{\mu})/\sqrt{2}\\
W_{\mu}^{\dagger} &= (W^1_{\mu}-i W^2_{\mu})/ \sqrt{2}\\
W_{\mu}^3 &= \cos \theta_W Z_{\mu} + \sin \theta_W A_{\mu}
\end{align*}
\end{flushleft}


\end{document}