\documentclass[11pt]{article}
\usepackage{graphicx}
\usepackage[utf8]{inputenc} 
\usepackage{amsmath}
\usepackage{cancel}
\usepackage{bbold}
\usepackage{color}
\usepackage{amsfonts}
\usepackage{mathtools}
\usepackage{braket}
\usepackage{float}
\usepackage{lscape}
\usepackage{multicol}
\usepackage{tikz-feynman}
\usepackage{tikz}
\usepackage{subcaption}
\usepackage{multicol}

\usepackage{geometry}
\geometry{legalpaper, margin=0.5in}

\begin{document}
\title{FYS4560 Project 2}
\author{Ingrid A V Holm}
\maketitle

\section{Di-lepton production in $e^+e^-$ in the Standard Model}


\begin{flushleft}
Possible feynman diagrams for the process
\begin{align*}
e^-e^+ \rightarrow l^-l^+
\end{align*}
\begin{figure}[H]
\centering
\begin{tikzpicture}
\begin{feynman}
\vertex (a) {\(e^-\)}; \vertex [below right=of a] (b) ; \vertex [below left=of b] (c) {\(e^+\)}; \vertex [right=of b] (s); \vertex [above right=of s] (s1) {\(l^+\)}; \vertex [below right=of s] (s2) {\(l^-\)}; 
\diagram{
(a) --[fermion] (b) --[fermion] (c); (b) --[photon, edge label={\(\gamma, Z\)}] (s); (s1) --[fermion] (s) --[fermion] (s2)
};
\end{feynman}
\end{tikzpicture}
\begin{tikzpicture}
\begin{feynman}[small]
\vertex (s1) {\(e^+\)} ; \vertex [below right=of s1] (s); \vertex [above right=of s] (s2) {\(e^+\)}; \vertex [below=of s] (p); \vertex [below right=of p] (p2) {\(e^-\)}; \vertex [below left=of p] (p1) {\(e^-\)};
\diagram{
(s2) --[fermion] (s) --[fermion] (s1), (p1) --[fermion] (p) --[fermion] (p2), (s) --[photon, edge label={\(\gamma, Z\)}] (p)
};
\end{feynman}
\end{tikzpicture}
\caption{Possible diagrams for di-lepton production.}
\label{fig:: feynman ee-ll}
\end{figure}
\end{flushleft}

\subsection{$e^- e^+ \rightarrow \mu^+ \mu^-$ through QED}
\begin{flushleft}
We find the expression for the matrix element from the Feynman diagram in Fig. (\ref{fig:: ee- mu mu}).
\begin{align*}
i \mathcal{M} &= \bar{v}^s (p_2) (-ie \gamma^{\mu}) u^{s'} (p_1) \big( \frac{-g_{\mu \nu}}{k^2} \big) \bar{u}^r(k_1) (-ie \gamma^{\nu})v^{r'}(k_2)\\
&= \frac{i e^2}{s} \Big( \bar{v}^s (p_2) \gamma^{\mu} u^{s'} (p_1) \Big) \Big( \bar{u}^r(k_1) \gamma_{\mu}v^{r'}(k_2) \Big)
\end{align*}
Which gives for the matrix element squared
\begin{align*}
|\mathcal{M}|^2 &= \frac{e^4}{k^4} \Big( \bar{v} (p_2) \gamma^{\mu} u (p_1) \bar{u}(p_1) \gamma^{\nu} v(p_2) \Big) \Big(\bar{u} (k_1) \gamma_{\mu} v(k_2) \bar{v} (k_2) \gamma_{\nu} u(k_1) \Big)
\end{align*}
Averaging over spins and taking the trace we find
\begin{align*}
\frac{1}{4} \sum_{spins} |\mathcal{M}|^2 &= \frac{e^4}{k^4} \text{tr} \Big[ (\cancel{p}_2 - m_e) \gamma^{\mu} (\cancel{p}_1 + m_e) \gamma^{\nu}\Big] \text{tr} \Big[(\cancel{k}_1 + m_m) \gamma_{\mu} (\cancel{k}_2 - m_m) \gamma_{\nu} \Big].
\end{align*}
Since $m_e<<m_m$, we can set $m_e = 0$. Traces of odd numbers of gamma matrices are zero, so we can reduce the number of terms to
\begin{align*}
\frac{1}{4} \sum_{spins} |\mathcal{M}|^2 &= \frac{e^4}{4 k^4} \text{tr} \Big[ p_{2 \rho} p_{1 \sigma} \gamma^{\rho}\gamma^{\mu} \gamma^{\sigma} \gamma^{\nu}\Big] \text{tr} \Big[k_1^{\rho'} k_2^{\sigma'}\gamma_{\rho'}\gamma_{\mu} \gamma_{\sigma'} \gamma_{\nu} -m_m^2 \gamma_{\mu} \gamma_{\nu} \Big]\\
&= \frac{e^4}{k^4} 4 \Bigg[ \Big(
p_2^{\mu}p_1^{\nu} - (p_1 \cdot p_2)g^{\mu \nu} + p_2^{\nu} p_1^{\mu}
\Big) \Big(k_{1\mu}k_{2\nu} - (k_2 \cdot k_1)g_{\mu \nu} + k_{1 \nu} k_{2 \mu} -m_m^2 g_{\mu \nu} \Big)\\
&= \frac{e^4}{k^4} 8 \Bigg[ (p_1 \cdot k_2)(p_2 \cdot k_1) 
 + (p_1 \cdot k_1)(p_2 \cdot k_2) -m_m^2 (p_1 \cdot p_2)\Bigg]\\
\end{align*}
\end{flushleft}

\subsubsection{Kinematics}
\begin{flushleft}
Assume the electron and muon momenta make an angle $\theta$ between them. Using the Mandelstam variables, and assuming $m_e$ we then get
\begin{align*}
k^2 &= (p_1 + p_2)^2 = 4E^2 = s\\
p_1 \cdot k_1 &= p_2 \cdot k_2 = E^2 - E|\textbf{k}| \cos \theta\\
p_1 \cdot k_2 &= p_2 \cdot k_1 = E^2 + E|\textbf{k}| \cos \theta
\end{align*}
Putting this into our expression we get
\begin{align*}
\frac{1}{4} \sum_{spins} |\mathcal{M}|^2 &= \frac{8e^4}{16s^2} \Big[E^2(E -\textbf{k} \cos \theta)^2 + E^2 (E +\textbf{k} \cos \theta)^2 + 2m_m^2 E^2 \Big]\\
&= e^4 \Big[ \Big(1 + \frac{m_m^2}{s} \Big) + \Big(1 - \frac{m_m^2}{s} \cos^2 \theta \Big) \Big]
\end{align*}
\end{flushleft}

\subsubsection{Differential cross section}
\begin{flushleft}
We can find the differential cross section for this process. The formula for differential cross section of two final-state particles in the center of mass frame is
\begin{align*}
\Big( \frac{d \sigma}{d \Omega} \Big)_{CM} = \frac{1}{2E_{\mathcal{A}}2 E_{\mathcal{B}} |v_{\mathcal{A}}-v_{\mathcal{B}}|} \frac{|\textbf{p}_1|}{(2 \pi)^2 4 E_{cm}}|\mathcal{M}(p_{\mathcal{A}}, p_{\mathcal{B}} \rightarrow p_1, p_2)|^2.
\end{align*}
For this specific calculation $|v_{\mathcal{A}}-v_{\mathcal{B}}|=2$, and $E_{\mathcal{A}}=E_{\mathcal{B}}= E_{cm}/2$, so we get
\begin{align*}
\frac{d \sigma}{d \Omega} &= \frac{1}{2 E_{cm}^2} \frac{|\textbf{k}|}{16 \pi^2 E_{cm}} \cdot \frac{1}{4} \sum_{spins} |\mathcal{M}|^2\\
&= \frac{\alpha}{4 s} \sqrt{1 - \frac{m_m^2}{s}} \Big[ \Big( 1 + \frac{m_m^2}{s}\Big) + \Big( 1 - \frac{m_m^2}{s}\Big) \cos^2 \theta \Big]
\end{align*}
We can rewrite the differential solid angle as $d \Omega = \sin \theta d \theta d \phi = - d \cos \theta d \phi$, so we get
\begin{align*}
\frac{d \sigma}{d \cos \theta} = - \frac{\alpha}{4 s} \sqrt{1 - \frac{m_m^2}{s}} \Big[ \Big( 1 + \frac{m_m^2}{s}\Big) + \Big( 1 - \frac{m_m^2}{s}\Big) \cos^2 \theta \Big] d \phi
\end{align*}
\end{flushleft}

\subsubsection{Cross section}
\begin{flushleft}
We can now find the total cross section by integrating over the angles
\begin{align*}
\sigma &= - \frac{\alpha}{4 s} \sqrt{1 - \frac{m_m^2}{s}} \int_{\Omega}   \Big[ \Big( 1 + \frac{m_m^2}{s}\Big) + \Big( 1 - \frac{m_m^2}{s}\Big) \cos^2 \theta \Big] d \phi d \cos \theta\\
&= 2 \pi \frac{\alpha}{4 s} \sqrt{1 - \frac{m_m^2}{s}}  \Big[ \Big( 1 + \frac{m_m^2}{s}\Big)\cos \theta + \frac{1}{3} \Big( 1 - \frac{m_m^2}{s}\Big) \cos^3 \theta \Big]_{-1}^{1}\\
&= 4 \pi \frac{\alpha}{4 s} \sqrt{1 - \frac{m_m^2}{s}}  \Big[ \Big( 1 + \frac{m_m^2}{s}\Big) + \frac{1}{3} \Big( 1 - \frac{m_m^2}{s}\Big) \Big]\\
\sigma &=\frac{4 \pi \alpha}{3 s}\sqrt{1 - \frac{m_m^2}{s}}  \Big[1 + \frac{1}{2}\frac{m_m^2}{s}\Big]\\
\end{align*}
\end{flushleft}

\pagebreak


\subsection{$e^+ e^- \rightarrow \mu^+ \mu^-$}
\begin{flushleft}
The feynman diagrams that contribute in the electroweak interaction are shown in Fig. (\ref{fig:: ee- mu mu}).
\begin{figure}[H]
\centering
\begin{tikzpicture}
\begin{feynman}[]
\vertex (a) {\(e^-\)}; \vertex [below right=of a] (b) ; \vertex [below left=of b] (c) {\(e^+\)}; \vertex [right=of b] (s); \vertex [above right=of s] (s1) {\(\mu^+\)}; \vertex [below right=of s] (s2) {\(\mu^-\)}; 
\diagram{
(a) --[fermion] (b) --[fermion] (c); (b) --[photon, edge label={\(\gamma\)}] (s); (s1) --[fermion] (s) --[fermion] (s2)
};
\end{feynman}
\end{tikzpicture}
\begin{tikzpicture}
\begin{feynman}
\vertex (a) {\(e^-\)}; \vertex [below right=of a] (b) {\(\mu\)}; \vertex [below left=of b] (c) {\(e^+\)}; \vertex [right=of b] (s) {\(\nu\)}; \vertex [above right=of s] (s1) {\(\mu^+\)}; \vertex [below right=of s] (s2) {\(\mu^-\)}; 
\diagram{
(a) --[fermion, momentum={\(p_1\)}] (b) --[fermion] (c); (c) -- [momentum={\(p_2\)}] (b); 
(b) --[photon, edge label={\( Z\)}] (s); 
(s1) --[fermion] (s) --[fermion, momentum={\(k_2\)}] (s2);
(s) --[, momentum={\(k_1\)}] (s1)
};
\end{feynman}
\end{tikzpicture}
\caption{Feynman diagrams for $e^+e^- \rightarrow \mu^+ \mu^-$.}
\label{fig:: ee- mu mu}
\end{figure}
The cross section for this process gets contributions from the purely electromagnetic ($\gamma$), the purely electroweak ($Z$) and the cross terms between them. We will calulate the terms separately. 
\end{flushleft}

\subsection{Purely electromagnetic interaction}

\begin{flushleft}
Here we can use the expression found earlier, but since we are working with a much larger energy scale (assume $E_e >> m_e$), we can set the muon mass to zero as well, $m_m = 0$, with yields the expression
\begin{align*}
\frac{1}{4} \sum_{spins} |\mathcal{M}|^2 
&= e^4 \big( 1 + \cos^2 \theta \big),
\end{align*}
which we can rewrite using $e^4 = (e^2)^2 = (4 \pi \alpha)^2$ to get
\begin{align*}
\frac{1}{4} \sum_{spins} |\mathcal{M}|^2 
&= 16 \pi^2 \alpha^2 \big( 1 + \cos^2 \theta \big).
\end{align*}
\end{flushleft}

\subsection{Purely electroweak contribution} 
\begin{flushleft}
We will here use the expressions
\begin{align}
g_v^f &= \frac{1}{2} T_f^3 - Q_f \sin^2 \theta_W,\\
g_A^f &= \frac{1}{2} T_f^3.
\end{align}
\end{flushleft}

\begin{center}
The expression for the matrix element from the feynman diagram in Fig. (\ref{fig:: ee- mu mu}) is
\end{center}
\begin{align*}
i\mathcal{M} &= \bar{v}^s(p_2) \big[i \frac{g}{\cos \theta_W} \gamma_{\mu} (g_v^f - g_A^f \gamma_5)\big] u^s(p_1) \Big(- \frac{ig_{\mu \nu}}{k^2 - M_Z^2 + i \epsilon} \Big) \bar{u}^s (k_2) \big[i \frac{g}{\cos \theta_W} \gamma_{\nu} (g_v^f - g_A^f \gamma_5)\big] v^s (k_1)\\
&= 
\frac{i g^2}{(k^2 - M_Z^2 + i \epsilon)\cos^2 \theta_W} 
\bar{v}^s(p_2) \gamma_{\mu} (g_v^f - g_A^f \gamma_5) u^s(p_1)  \bar{u}^s (k_2) \gamma^{\mu} (g_v^f - g_A^f \gamma_5) v^s (k_1)\\
\end{align*}
\begin{align*}
|\mathcal{M}|^2 &= \frac{g^4}{(k^2 - M_Z^2)^2\cos^4 \theta_W} 
\Big( \bar{v}^s(p_2) \gamma_{\mu} (g_v^f - g_A^f \gamma_5) u^s(p_1) 
\bar{u}^s(p_1) \gamma_{\nu} (g_v^f - g_A^f \gamma_5) v^s(p_2) 
\Big)\\
& \times
\Big(
 \bar{u}^s (k_2) \gamma^{\mu} (g_v^f - g_A^f \gamma_5) v^s (k_1)
\bar{v}^s (k_1) \gamma^{\nu} (g_v^f - g_A^f \gamma_5) u^s (k_2) 
  \Big)
\end{align*}
Take the trace and get (where $A^2 = \frac{g^4}{(k^2 - M_Z^2)^2\cos^4 \theta_W} $), and since $M_z >> m_e, m_{\mu}$, set $m_e = m_{\mu} \simeq 0$
\begin{align*}
\frac{1}{4} \sum_{spins} |\mathcal{M}|^2 &= \frac{1}{4} A^2 \text{tr}
\Big[ (\cancel{p}_2 -m_e)  
\gamma_{\mu} (g_v^e - g_A^e \gamma_5) (\cancel{p}_1 + m_e) 
\gamma_{\nu} (g_v^e - g_A^e \gamma_5)\Big]\\
& \times \text{tr} \Big[
(\cancel{k}_2 + m_{\mu}) \gamma^{\mu} (g_v^l - g_A^l \gamma_5) (\cancel{k}_1 - m_{\mu}) \gamma^{\nu} (g_v^l - g_A^l \gamma_5) 
\Big]\\
\text{(set mass to zero)}
&= \frac{1}{4} A^2 \text{tr}
\Big[ \cancel{p}_2 
\gamma_{\mu} (g_v^e - g_A^e \gamma_5) \cancel{p}_1 
\gamma_{\nu} (g_v^e - g_A^e \gamma_5)\Big]
\text{tr} \Big[
\cancel{k}_2 \gamma^{\mu} (g_v^l - g_A^l \gamma_5) \cancel{k}_1 \gamma^{\nu} (g_v^l - g_A^l \gamma_5) 
\Big]\\
&= A^2 \text{tr}
\Big[ (g_v^e)^2 \cancel{p}_2 
\gamma_{\mu} 
\cancel{p}_1 
\gamma_{\nu} 
+
(g_A^e)^2
\cancel{p}_2 
\gamma_{\mu} \gamma_5
 \cancel{p}_1 
\gamma_{\nu} \gamma_5
- g_v^e g_A^e 
\cancel{p}_2 
\gamma_{\mu} \cancel{p}_1 
\gamma_{\nu} \gamma_5
- g_v^e g_A^e \cancel{p}_2 
\gamma_{\mu} \gamma_5 \cancel{p}_1 
\gamma_{\nu} g_v^e)
\Big] \\
& \times
\text{tr} \Big[
(g_v^l)^2
\cancel{k}_2 \gamma^{\mu} 
 \cancel{k}_1 \gamma^{\nu} 
+(g_A^l)^2
\cancel{k}_2 \gamma^{\mu} \gamma_5
 \cancel{k}_1 \gamma^{\nu} \gamma_5 
- g_v^l g_A^l 
\cancel{k}_2 
\gamma_{\mu} \cancel{k}_1 
\gamma_{\nu} \gamma_5
- g_v^l g_A^l \cancel{k}_2 
\gamma_{\mu} \gamma_5 \cancel{k}_1 
\gamma_{\nu}\gamma_5)
\Big] \\
&= \frac{1}{4} A^2 \text{tr}
\Big[  
\big((g_A^e)^2 + (g_v^e)^2\big)
\cancel{p}_2 
\gamma_{\mu} 
 \cancel{p}_1 
\gamma_{\nu} 
- 2 g_v^e g_A^e 
\cancel{p}_2 
\gamma_{\mu} \cancel{p}_1 
\gamma_{\nu} \gamma_5 \Big]\\
& \times
\text{tr} \Big[
\big( (g_v^l)^2 +(g_A^l)^2 \big)
\cancel{k}_2 \gamma^{\mu} 
 \cancel{k}_1 \gamma^{\nu} 
- 2 g_v^l g_A^l 
\cancel{k}_2 
\gamma_{\mu} \cancel{k}_1 
\gamma_{\nu} \gamma_5
\Big] \\
&= \frac{1}{4} A^2 \text{tr}
\Big[
 ((g_v^e)^2 + (g_A^e)^2)p_2^{\rho} p_1^{\sigma} \gamma_{\rho} 
\gamma_{\mu} 
 \gamma_{\sigma}
\gamma_{\nu} 
- 2 g_v^e g_A^e 
p_2^{\rho}  p_1^{\sigma} \gamma_{\rho} \gamma_{\mu}\gamma_{\sigma} \gamma_{\nu} \gamma_5
\Big]\\
&\times \text{tr} \Big[
((g_v^l)^2 + (g_A^l)^2) k_{2 \rho} k_{1 \sigma} \gamma^{\rho} \gamma^{\mu} 
 \gamma^{\sigma} \gamma^{\nu}
 - 2 g_v^l g_A^l 
k_2^{ \rho} k_1^{ \sigma} \gamma_{\rho}
\gamma_{\mu}  \gamma_{\sigma}
\gamma_{\nu} \gamma_5 
\Big]\\
\end{align*}

\begin{center}
Now use the trace identities for the gamma matrices
\begin{align}
&\text{tr} (\gamma^{\mu} \gamma^{\nu} \gamma^{\rho} \gamma^{\sigma}) = 4 (g^{\mu\nu} g^{\rho \sigma} - g^{\mu \rho}g^{\nu \sigma} + g^{\mu \sigma}g^{\nu \rho})\\
& \text{tr} (\gamma^{\mu} \gamma^{\nu} \gamma^{\rho} \gamma^{\sigma} \gamma^5) = -4i \epsilon^{\mu \nu \rho \sigma}
\end{align}
\end{center}
\begin{align*}
\frac{1}{4} \sum_{spins} |\mathcal{M}|^2 &= \frac{1}{4} 16 A^2 \Big[
 ((g_v^e)^2 + (g_A^e)^2)p_2^{\rho} p_1^{\sigma} \big(
g_{\rho \mu} g_{\sigma \nu} 
- g_{\rho \sigma} g_{\mu \nu}
+ g_{\rho \nu} g_{\mu \sigma}
\big)
- 2 g_v^e g_A^e p_2^{\rho}  p_1^{\sigma} 
\epsilon_{\rho \mu \sigma \nu}
\Big]\\
&\times \Big[
((g_v^l)^2 + (g_A^l)^2) k_{2 \rho} k_{1 \sigma} 
\big(
g^{\rho \mu} g^{\sigma \nu} 
- g^{\rho \sigma} g^{\mu \nu}
+ g^{\rho \nu} g^{\mu \sigma}
\big)
 - 2 g_v^l g_A^l 
k_{2 \rho} k_{1 \sigma} 
\epsilon^{\rho \mu \sigma \nu}
\Big]\\
&= \frac{1}{4} 16 A^2 \Big[
 \{((g_v^e)^2 + (g_A^e)^2)
 ((g_v^l)^2 + (g_A^l)^2)\} \cdot
 p_2^{\rho} p_1^{\sigma} \big(
g_{\rho \mu} g_{\sigma \nu} 
- g_{\rho \sigma} g_{\mu \nu}
+ g_{\rho \nu} g_{\mu \sigma}
\big) k_{2 \rho} k_{1 \sigma} 
\big(
g^{\rho \mu} g^{\sigma \nu} 
- g^{\rho \sigma} g^{\mu \nu}
+ g^{\rho \nu} g^{\mu \sigma}
\big)\\
 &- \{2 ((g_v^e)^2 + (g_A^e)^2) g_v^l g_A^l \} \cdot
 p_2^{\rho} p_1^{\sigma} \big(
g_{\rho \mu} g_{\sigma \nu} 
- g_{\rho \sigma} g_{\mu \nu}
+ g_{\rho \nu} g_{\mu \sigma}
\big)
k_{2 \rho} k_{1 \sigma} 
\epsilon^{\rho \mu \sigma \nu}
)\\
&- \{2 ((g_v^l)^2 + (g_A^l)^2) g_v^e g_A^e \} \cdot
p_2^{\rho}  p_1^{\sigma} 
\epsilon_{\rho \mu \sigma \mu}
 k_{2 \rho} k_{1 \sigma} 
\big(
g^{\rho \mu} g^{\sigma \nu} 
- g^{\rho \sigma} g^{\mu \nu}
+ g^{\rho \nu} g^{\mu \sigma}
\big)\\
& +\{4  g_v^e g_A^e g_v^l g_A^l  \} \cdot
p_2^{\rho}  p_1^{\sigma} 
\epsilon_{\rho \mu \sigma \nu} 
k_{2 \rho} k_{1 \sigma} 
\epsilon^{\rho \mu \sigma \nu}
\Big]\\
&= \frac{1}{4} 16 A^2 \Big[
 \{((g_v^e)^2 + (g_A^e)^2)  ((g_v^l)^2 + (g_A^l)^2)\} \cdot
\big( p_{2 \mu} p_{1 \nu} 
- p_{2 \sigma} p_1^{ \sigma} g_{\mu \nu}
+ p_{2 \nu} p_{1 \mu} \Big)\Big( k_2^{\mu} k_1^{\nu}
- k_2^{\sigma} k_{1 \sigma} g^{\mu \nu}
+ k_2^{\nu} k_1^{\mu}\Big)\\
 &-
  \{2 ((g_v^e)^2 + (g_A^e)^2) g_v^l g_A^l \} \cdot
 \big(
 p_{2\mu} p_{1 \nu} - p_{2\sigma} p_1^{ \sigma} g_{\mu\nu}
+ p_{2\nu} p_{1 \mu} \big)
k_{2 \rho} k_{1 \sigma} \epsilon^{\rho \mu \sigma \nu})\\
&- 
\{2 ((g_v^l)^2 + (g_A^l)^2) g_v^e g_A^e \} \cdot
p_2^{\rho}  p_1^{\sigma} \epsilon_{\rho \mu \sigma \mu}
\big( k_2^{ \mu} k_1^{ \nu}  - k_2^{ \sigma} k_{1 \sigma} g^{\mu \nu} + k_2^{ \nu} k_1^{ \mu}\big)+
4! \{4  g_v^e g_A^e g_v^l g_A^l  \} \cdot
p_2^{\rho}  k_{2 \rho} p_1^{\sigma} k_{1 \sigma} 
\Big]\\
\end{align*}
\begin{center}
Where we've used
\begin{align*}
\epsilon_{\rho \mu \sigma \mu} \epsilon^{\rho \mu \sigma \mu} = n! = 4! = 24. 
\end{align*}
\end{center}
\begin{align*}
&= \frac{1}{4} 16 A^2 \Big[
 \{((g_v^e)^2 + (g_A^e)^2)  ((g_v^l)^2 + (g_A^l)^2)\} \cdot
\Big[ 2(p_2 \cdot k_2) (p_1 \cdot k_1) + 2(p_1 \cdot k_2) (p_2 \cdot k_1) \Big]\\
 &-
\{2 ((g_v^e)^2 + (g_A^e)^2) g_v^l g_A^l \} \cdot
\Big( \epsilon^{\rho \mu \sigma \nu}  \big(
p_{2\mu} p_{1 \nu} - p_{2\sigma} p_1^{ \sigma} g_{\mu\nu}
+ p_{2\nu} p_{1 \mu} \big) k_{2 \rho} k_{1 \sigma} 
+ \epsilon^{\rho \mu \sigma \nu} p_{2 \rho}  p_{1 \sigma}  \big(k_{2 \mu} k_{1 \nu}  - k_{2 \sigma} k_1^{ \sigma} g_{\mu \nu} + k_{2 \nu} k_{1 \mu}\big) \Big)\\
& +
4! \{4  g_v^e g_A^e g_v^l g_A^l  \} \cdot
(p_2 \cdot k_2) (p_1 \cdot k_1)
\Big]\\
&= I + II + III
\end{align*}

\subsubsection{Second term ($II$)}
\begin{center}
Take a look at the second term, and notice that $\epsilon^{\mu\nu \rho \sigma}$ is zero if any two indices are equal, so $\epsilon^{\mu \nu \rho \sigma} g_{\mu \nu}=0$, so we get
\begin{flushleft}
\begin{align*}
&  \epsilon^{\rho \mu \sigma \nu}  \big(
p_{2\mu} p_{1 \nu} - p_{2\sigma} p_1^{ \sigma} g_{\mu\nu}
+ p_{2\nu} p_{1 \mu} \big) k_{2 \rho} k_{1 \sigma} 
+ \epsilon^{\rho \mu \sigma \nu} p_{2 \rho}  p_{1 \sigma}  \big(k_{2 \mu} k_{1 \nu}  - k_{2 \sigma} k_1^{ \sigma} g_{\mu \nu} + k_{2 \nu} k_{1 \mu}\big) \\
&= \epsilon^{\rho \mu \sigma \nu}  \big(
p_{2\mu} p_{1 \nu}k_{2 \rho} k_{1 \sigma} 
+ p_{2\nu} p_{1 \mu} k_{2 \rho} k_{1 \sigma} \big) + \epsilon^{\rho \mu \sigma \nu} \big(
 p_{2 \rho}  p_{1 \sigma} k_{2 \mu} k_{1 \nu}  + p_{2 \rho}  p_{1 \sigma}  k_{2 \nu} k_{1 \mu}\big) \\
&= \epsilon^{\rho \mu \sigma \nu}  \big(
p_{2\mu} p_{1 \nu}k_{2 \rho} k_{1 \sigma} 
+ p_{2\nu} p_{1 \mu} k_{2 \rho} k_{1 \sigma} \big) + \epsilon^{\rho \mu  \sigma \nu } \big(
p_{2 \mu}  p_{1 \nu} k_{2 \rho} k_{1 \sigma}  + p_{2 \mu}  p_{1 \nu}  k_{2 \sigma} k_{1 \rho}\big) \\
&= 2 \big( \epsilon^{\rho \mu  \sigma \nu } 
p_{2 \mu}  p_{1 \nu} k_{2 \rho} k_{1 \sigma}  + \epsilon^{\rho \mu  \sigma \nu } p_{2 \mu}  p_{1 \nu}  k_{2 \sigma} k_{1 \rho}\big) \\
&= 2 \big( \epsilon^{\rho \mu  \sigma \nu } 
p_{2 \mu}  p_{1 \nu} k_{2 \rho} k_{1 \sigma}  
- \epsilon^{\rho \mu  \sigma \nu } p_{2 \mu}  p_{1 \nu}  k_{2 \rho} k_{1 \sigma}\big) = \underline{0} \\
\end{align*}
\end{flushleft}
\end{center}


\subsubsection{First term ($I$)}
\begin{align*} 
 p_2^{\rho} p_1^{\sigma} 4\big(
g_{\rho \mu} g_{\sigma \nu} - g_{\rho \sigma} g_{\mu \nu} + g_{\rho \nu} g_{\mu \sigma}
\Big)& 4k_{2\rho} k_{1\sigma}\Big( g^{\rho \mu} g^{\sigma \nu} - g^{\rho \sigma} g^{\mu \nu}
+ g^{\rho \nu} g^{\mu \sigma} \Big)\\
&= \big( p_{2 \mu} p_{1 \nu} - p_{2 \sigma} p_1^{ \sigma} g_{\mu \nu} + p_{2 \nu} p_{1 \mu} \Big)
\Big(k_2^{\mu} k_1^{\nu} - k_2^{\sigma} k_{1 \sigma} g^{\mu \nu} + k_2^{\nu} k_1^{\mu}\Big)\\ &= 
\Big[ (p_2 \cdot k_2) (p_1 \cdot k_1) - (k_2 \cdot k_1) (p_2 \cdot p_1) + (p_1 \cdot k_2) (p_2 \cdot k_1)\\ & - (p_2 \cdot p_1) (k_2 \cdot k_1) 
+ 4(k_2 \cdot k_1) (p_2 \cdot p_1) - (p_2 \cdot p_1) (k_2 \cdot k_1)\\ &  +(k_2 \cdot p_1) (p_2 \cdot k_1)
- (k_2 \cdot k_1) (p_2 \cdot p_1) + (p_2 \cdot k_2) (p_1 \cdot k_1) \Big]\\ &= \Big[ 2(p_2 \cdot k_2) (p_1 \cdot k_1)  + 2(p_1 \cdot k_2) (p_2 \cdot k_1) \Big]\\
\end{align*}

\subsubsection*{Back to $|\mathcal{M}|^2$}
\begin{center}
\begin{align*}
\frac{1}{4} \sum_{spins}| \mathcal{M}|^2 &= \frac{1}{4} 16 A^2 \Big[
 \{2 ((g_v^e)^2 + (g_A^e)^2)  ((g_v^l)^2 + (g_A^l)^2) + 4 \cdot 4! \cdot g_v^e g_A^e g_v^l g_A^l\} \cdot
(p_2 \cdot k_2) (p_1 \cdot k_1)\\
& + 2 \{((g_v^e)^2 + (g_A^e)^2)  ((g_v^l)^2 + (g_A^l)^2)\}(p_1 \cdot k_2) (p_2 \cdot k_1) \Big]\\
&= \frac{1}{4} 32 \frac{g^4}{(s - M_Z^2)^2\cos^4 \theta_W} \Big[
 \{((g_v^e)^2 + (g_A^e)^2)  ((g_v^l)^2 + (g_A^l)^2) + 48 \cdot g_v^e g_A^e g_v^l g_A^l\} \cdot
(p_2 \cdot k_2) (p_1 \cdot k_1)\\
& +  \{((g_v^e)^2 + (g_A^e)^2)  ((g_v^l)^2 + (g_A^l)^2)\}(p_1 \cdot k_2) (p_2 \cdot k_1) \Big]\\
\end{align*}
\end{center}

\subsection*{Kinematics}
\begin{flushleft}
We work in the center of mass-frame. Since $|\textbf{k}| = \sqrt{E^2 - m_{\mu}^2} \simeq \sqrt{E^2} = E$, we get
\begin{align*}
s &= k^2 = (p_1 + p_2)^2 = (k_1 + k_2)^2 = 4E^2\\
(p_1 \cdot k_1) &= (p_2 \cdot k_2) = E^2 - E|\textbf{k}| \cos \theta \simeq E^2 (1 - \cos \theta) = \frac{1}{4} s (1 - \cos \theta)\\
(p_1 \cdot k_2) &= (p_2 \cdot k_1) = E^2 + E|\textbf{k}| \cos \theta \simeq E^2(1+ \cos \theta) = \frac{1}{4} s (1 + \cos \theta)
\end{align*}
Putting this into the expression we get
\begin{align*}
\frac{1}{4} \sum_{spins}| \mathcal{M}|^2 = 
\frac{1}{4} 32 \frac{g^4}{(s - M_Z^2)^2\cos^4 \theta_W} &\Big[
 \{((g_v^e)^2 + (g_A^e)^2)  ((g_v^l)^2 + (g_A^l)^2) + 48 \cdot g_v^e g_A^e g_v^l g_A^l\} \cdot
(\frac{1}{4} s (1 - \cos \theta))^2\\
& +  \{((g_v^e)^2 + (g_A^e)^2)  ((g_v^l)^2 + (g_A^l)^2)\}(\frac{1}{4} s (1 + \cos \theta))^2 \Big]\\
= \frac{1}{4} 4 \frac{g^4s^2}{(s - M_Z^2)^2\cos^4 \theta_W} &\Big[
 \{((g_v^e)^2 + (g_A^e)^2)  ((g_v^l)^2 + (g_A^l)^2) \} \cdot
(1  + \cos^2 \theta)+ 24  g_v^e g_A^e g_v^l g_A^l  (1 -  \cos \theta)^2 \Big]\\
\end{align*}
We know that muons and electrons behave in the exact same way, only their mass is different, so we set $g_v^e = g_v^l$, $g_A^e=g_A^l$
\begin{align*}
\frac{1}{4} \sum_{spins}| \mathcal{M}|^2 &= \frac{1}{4} 4 \frac{g^4s^2}{(s - M_Z^2)^2\cos^4 \theta_W} &\Big[
((g_v^e)^2 + (g_A^e)^2)^2 \cdot
(1  + \cos^2 \theta)+ 24  (g_v^e)^2 (g_A^e)^2 (1 -  \cos \theta)^2 \Big]\\
&= \frac{1}{4} 4 \frac{g^4s^2}{(s - M_Z^2)^2\cos^4 \theta_W} &\Big[
((g_v^e)^2 + (g_A^e)^2)^2 \cdot
(1  + \cos^2 \theta)+ 24  (g_v^e)^2 (g_A^e)^2 (1 -  2\cos \theta + \cos^2 \theta) \Big]\\
&= \frac{1}{4} 4 \frac{g^4s^2}{(s - M_Z^2)^2\cos^4 \theta_W} &\Big[
[((g_v^e)^2 + (g_A^e)^2)^2 + 24 (g_v^e)^2 (g_A^e)^2] \cdot
(1  + \cos^2 \theta)- 48  (g_v^e)^2 (g_A^e)^2 2\cos \theta \Big]\\
\end{align*}
Here we can use that $g^4/\cos^4 \theta_w = (e/\sin \theta_W)^4/\cos^4 \theta_W = e^4/(\sin \theta_W \cos_W)^4 = e^4/(1/2 \sin 2\theta_W)^4 = e^4 (M_Z^2)^2G^2 2/(\alpha^2 \pi^2)$ to rewrite this expression
\begin{align*}
\frac{1}{4} \sum_{spins}| \mathcal{M}|^2 &= 
8\frac{1}{4} \frac{e^4 G^2 s^2}{\alpha^2 \pi^2} \Big( \frac{M_Z^2}{s -M_Z^2} \Big)^2 \Big[
[((g_v^e)^2 + (g_A^e)^2)^2 + 24 (g_v^e)^2 (g_A^e)^2] \cdot
(1  + \cos^2 \theta)- 48  (g_v^e)^2 (g_A^e)^2 2\cos \theta \Big]\\
&= 32 G^2 s^2 \Big( \frac{M_Z^2}{s -M_Z^2} \Big)^2 \Big[
[((g_v^e)^2 + (g_A^e)^2)^2 + 24 (g_v^e)^2 (g_A^e)^2] \cdot
(1  + \cos^2 \theta)- 96 (g_v^e)^2 (g_A^e)^2 \cos \theta \Big]\\
\end{align*}

\end{flushleft}

\pagebreak

\subsection{Cross terms}
\begin{flushleft}
We get the cross terms from the expression
\begin{align*}
|\mathcal{M}|^2 &= (i\mathcal{M}_{\gamma} + i\mathcal{M}_{Z})(-i\mathcal{M}_{\gamma}^* -i \mathcal{M}_{Z}^*)\\
&= |\mathcal{M}_{\gamma}|^2 + \mathcal{M}_{\gamma} \mathcal{M}_Z^* + \mathcal{M}_Z \mathcal{M}_{\gamma}^* + |\mathcal{M}_Z|^2.
\end{align*}
We use the expressions we've found for $\mathcal{M}_{Z/\gamma}$, and write out the cross terms, which we will call $\mathcal{M}_{\times} = \mathcal{M}_{\gamma} \mathcal{M}_Z^* + \mathcal{M}_Z \mathcal{M}_{\gamma}^*$
\begin{align*}
\mathcal{M}_{\times}
 &= \Bigg(\frac{i e^2}{s} \Big( \bar{v}^s (p_2) \gamma^{\mu} u^{s'} (p_1) \Big) \Big( \bar{u}^r(k_1) \gamma_{\mu}v^{r'}(k_2) \Big) \Bigg)\\
& \times \Bigg(\frac{i g^2}{(k^2 - M_Z^2 + i \epsilon)\cos^2 \theta_W} 
\bar{v}^s(p_2) \gamma_{\nu} (g_v^f - g_A^f \gamma_5) u^s(p_1)  \bar{u}^s (k_1) \gamma^{\nu} (g_v^f - g_A^f \gamma_5) v^s (k_2)\Bigg)^*\\
&+ \Bigg(\frac{i g^2}{(k^2 - M_Z^2 + i \epsilon)\cos^2 \theta_W} 
\bar{v}^s(p_2) \gamma_{\nu} (g_v^f - g_A^f \gamma_5) u^s(p_1)  \bar{u}^s (k_1) \gamma^{\nu} (g_v^f - g_A^f \gamma_5) v^s (k_2)\Bigg)\\
&\times \Bigg(\frac{i e^2}{s} \Big( \bar{v}^s (p_2) \gamma^{\mu} u^{s'} (p_1) \Big) \Big( \bar{u}^r(k_1) \gamma_{\mu}v^{r'}(k_2) \Big) \Bigg)^*\\
%Ny linje--------------------------------
%----------------------------------------
&= \frac{e^2 g^2}{s(k^2 - M_Z^2 + i \epsilon)\cos^2 \theta_W} \Bigg[
 \Big( \bar{v} (p_2) \gamma^{\mu} u (p_1)  \bar{u}(p_1) \gamma_{\nu} (g_v^f - g_A^f \gamma_5)v(p_2) \Big) \Big( \bar{u}(k_1) \gamma_{\mu}v(k_2)  \bar{v} (k_2) \gamma^{\nu} (g_v^f - g_A^f \gamma_5)  u(k_1) \Big) \\
&+ \big(
\bar{v} (p_2) \gamma_{\nu} (g_v^f - g_A^f \gamma_5) u (p_1) 
\bar{u} (p_1) \gamma^{\mu}  v (p_2) \big)
\big(
 \bar{u} (k_1) \gamma^{\nu} (g_v^f - g_A^f \gamma_5) v (k_2) \bar{v}(k_2) \gamma_{\mu} u(k_1)\big) \Bigg]\\
\end{align*}

Introduce the coefficient  $B = \frac{e^2 g^2}{s(k^2 - M_Z^2 + i \epsilon)\cos^2 \theta_W}$ and take the trace and average over spins, also note that we can move $\gamma^5 \cancel{p}\gamma^{\mu}= -\cancel{p} \gamma^5 \gamma^{\mu} = \cancel{p}\gamma^{\mu} \gamma^5$
\begin{align*}
\frac{1}{4B} \sum_{spins} \mathcal{M}_{\times} &= \frac{1}{4} \text{tr} \Big[
 (\cancel{p}_2 - m_e) \gamma^{\mu} (\cancel{p}_1 + m_e) \gamma_{\nu} (g_v^f - g_A^f \gamma_5) \Big] \text{tr} \Big[(\cancel{k}_1 + m_m) \gamma_{\mu} (\cancel{k}_2 - m_m) \gamma^{\nu} (g_v^f - g_A^f \gamma_5) \Big] \\
&+ \text{tr} \Big[
(\cancel{p}_2 - m_e) \gamma_{\nu}(\cancel{p}_1 + m_e) \gamma^{\mu} (g_v^f - g_A^f \gamma_5)  \Big]
\text{tr}
\Big[
(\cancel{k}_1 +m_m) \gamma^{\nu}  (\cancel{k}_2 -m_m) \gamma_{\mu} (g_v^f - g_A^f \gamma_5) \Big] \\
&= \frac{1}{4} 2 \text{tr} \Big[
 (\cancel{p}_2 - m_e) \gamma^{\mu} (\cancel{p}_1 + m_e) \gamma_{\nu} (g_v^f - g_A^f \gamma_5) \Big] \text{tr} \Big[(\cancel{k}_1 + m_m) \gamma_{\mu} (\cancel{k}_2 - m_m) \gamma^{\nu} (g_v^f - g_A^f \gamma_5) \Big]
\end{align*}
We can now set the muon- and electron masses to zero
\begin{align*}
\frac{1}{8B} \sum_{spins} \mathcal{M}_{\times} &= \frac{1}{4} \text{tr} \Big[ p_{2 \rho} p_{1 \sigma}\gamma^{\rho}
\gamma^{\mu} \gamma^{\sigma} \gamma^{\nu} (g_v^f - g_A^f \gamma_5) \Big] \text{tr} \Big[ k_1^{\rho} k_2^{\sigma} \gamma_{\rho} \gamma_{\mu} \gamma_{\sigma} \gamma_{\nu} (g_v^f - g_A^f \gamma_5) \Big] \\
&= \frac{1}{4} \Big( p_{2 \rho} p_{1 \sigma} \big[ g_v^f(g^{\rho \mu} g^{\sigma \nu} - g^{\rho \sigma}g^{\mu \nu} + g^{\rho \nu}g^{\mu \sigma}) - g_A^f \epsilon^{\rho \mu \sigma \nu} \big] \Big)
\Big( k_1^{\rho} k_2^{\sigma}  \big[ g_v^f (g_{\rho \mu} g_{\sigma \nu} - g_{\rho \sigma}g_{\mu \nu} + g_{\rho \nu}g_{\mu \sigma})- g_A^f \epsilon_{\rho \mu \sigma \nu}\big] \Big) \\
&= \frac{1}{4} \Big( g_v^f(p_2^{\mu} p_1^{\nu} - (p_1 \cdot p_2)g^{\mu \nu} + p_2^{\nu} p_1^{\mu}) - g_A^f p_{2 \rho} p_{1 \sigma}\epsilon^{\rho \mu \sigma \nu}  \Big)
\Big(  g_v^f (k_{1 \mu} k_{2 \nu} - (k_1 \cdot k_2)g_{\mu \nu} + k_{1 \nu} k_{2 \mu})- g_A^f k_1^{\rho} k_2^{\sigma} \epsilon_{\rho \mu \sigma \nu} \Big) \\
&= \frac{1}{4}
g_v^f(p_2^{\mu} p_1^{\nu} - (p_1 \cdot p_2)g^{\mu \nu} + p_2^{\nu} p_1^{\mu})
 g_v^f (k_{1 \mu} k_{2 \nu} - (k_1 \cdot k_2)g_{\mu \nu} + k_{1 \nu} k_{2 \mu}) \\
&- g_A^f g_v^f \Big( p_{2 \rho} p_{1 \sigma}\epsilon^{\rho \mu \sigma \nu}
(k_{1 \mu} k_{2 \nu} - (k_1 \cdot k_2)g_{\mu \nu} + k_{1 \nu} k_{2 \mu}) +(p_2^{\mu} p_1^{\nu} - (p_1 \cdot p_2)g^{\mu \nu} + p_2^{\nu} p_1^{\mu})
k_1^{\rho} k_2^{\sigma} \epsilon_{\rho \mu \sigma \nu} \Big) \\
&+ g_A^f p_{2 \rho} p_{1 \sigma}\epsilon^{\rho \mu \sigma \nu}g_A^f k_1^{\rho} k_2^{\sigma} \epsilon_{\rho \mu \sigma \nu} \\
&= \frac{1}{4} 2 (g_v^f)^2\Big[(p_2 \cdot k_2) (p_1 \cdot k_1)  + (p_1 \cdot k_2) (p_2 \cdot k_1) \Big] 
+ 4! (g_A^f)^2 (p_1 \cdot k_2)(p_2 \cdot k_1) \\
\end{align*}
\end{flushleft}

\pagebreak

\subsubsection*{Kinematics}
\begin{flushleft}
Using the kinematics derived in the previous calculation, we can simplify this expression
\begin{align*}
\frac{1}{4} \sum_{spins} |\mathcal{M}|_{\times} &= \frac{1}{4} 4 B \Big((g_v^f)^2\Big[\frac{1}{16}s^2(1- \cos \theta)^2  + \frac{1}{16}s^2(1+ \cos \theta)^2 \Big] 
+ 12 (g_A^f)^2 \frac{1}{16}s^2(1+ \cos \theta)^2 \Big)\\
&= \frac{1}{4} \frac{1}{2} B s^2 \Big((g_v^f)^2 \Big[1+ \cos^2 \theta\Big] 
+ 6 (g_A^f)^2 (1+ \cos \theta)^2 \Big)\\
&= \frac{1}{4} \frac{1}{2} \frac{e^2 g^2}{s(s - M_Z^2)\cos^2 \theta_W} s^2 \Big((g_v^f)^2 \Big[1+ \cos^2 \theta\Big] 
+ 6 (g_A^f)^2 (1+ \cos \theta)^2 \Big)\\
&= \frac{1}{4} \frac{1}{2} \frac{e^2 g^2}{s(s - M_Z^2)\cos^2 \theta_W} s^2 
\Big([(g_v^f)^2 + 6 (g_A^f)^2] \Big[1+ \cos^2 \theta\Big] 
+ 12 (g_A^f)^2 \cos \theta \Big)\\
&= \frac{1}{4} \frac{1}{2} \frac{e^4}{s(s - M_Z^2)\cos^2 \theta_W \sin^2 \theta_W} s^2 
\Big([(g_v^f)^2 + 6 (g_A^f)^2] \Big[1+ \cos^2 \theta\Big] 
+ 12 (g_A^f)^2 \cos \theta \Big)\\
\end{align*}
We can now use that $\cos^{-2} \theta_W \sin^{-2} \theta_W = (\cos \theta_W \sin \theta_W)^{-2} = (1/2 \sin 2 \theta_W)^{-2} =M_Z^2G \sqrt{2}/(\alpha \pi)$, along with $e^4 = (e^2)^2 = (4 \pi \alpha)^2 = 16 \pi^2 \alpha^2$, to get
\begin{align*}
\frac{1}{4} \sum_{spins} |\mathcal{M}|_{\times} &=   \frac{2 \pi^2 \alpha^2 M_Z^2 G \sqrt{2}}{s(s - M_Z^2)\alpha \pi} s^2 
\Big([(g_v^f)^2 + 6 (g_A^f)^2] \Big[1+ \cos^2 \theta\Big] 
+ 12 (g_A^f)^2 \cos \theta \Big)\\
&= \frac{4}{\sqrt{2}} \pi \alpha G s \frac{M_Z^2 }{(s - M_Z^2)} 
\Big([(g_v^f)^2 + 6 (g_A^f)^2] \Big[1+ \cos^2 \theta\Big] 
+ 12 (g_A^f)^2 \cos \theta \Big)\\
\end{align*}
\end{flushleft}

\pagebreak

\subsection{Combining the terms}
\begin{flushleft}
We can now combine the matrix element terms
\begin{align*}
\frac{1}{4} \sum_{spins} |\mathcal{M}|^2 =&
16 \pi^2 \alpha^2 \big( 1 + \cos^2 \theta \big)\\
&+ 32 G^2 s^2 \Big( \frac{M_Z^2}{s -M_Z^2} \Big)^2 \Big[
[((g_v^e)^2 + (g_A^e)^2)^2 + 24 (g_v^e)^2 (g_A^e)^2] \cdot
(1  + \cos^2 \theta)- 96 (g_v^e)^2 (g_A^e)^2 \cos \theta \Big]\\
&+ \frac{4}{\sqrt{2}} \pi \alpha G s \frac{M_Z^2 }{(s - M_Z^2)} 
\Big([(g_v^f)^2 + 6 (g_A^f)^2] \Big[1+ \cos^2 \theta\Big] 
+ 12 (g_A^f)^2 \cos \theta \Big)\\
=& \Bigg[16 \pi^2 \alpha^2
+ 32 G^2 s^2 \Big( \frac{M_Z^2}{s -M_Z^2} \Big)^2
[((g_v^e)^2 + (g_A^e)^2)^2 + 24 (g_v^e)^2 (g_A^e)^2]\\
& + \frac{4}{\sqrt{2}} \pi \alpha G s \frac{M_Z^2 }{(s - M_Z^2)} [(g_v^f)^2 + 6 (g_A^f)^2]
 \Bigg] (1 + \cos^2 \theta)\\
&+ \Bigg[32 G^2 s^2 \Big( \frac{M_Z^2}{s -M_Z^2} \Big)^2 (- 96 (g_v^e)^2 (g_A^e)^2)
+ \frac{4}{\sqrt{2}} \pi \alpha G s \frac{M_Z^2 }{(s - M_Z^2)} 12 (g_A^f)^2
  \Bigg] \cos \theta\\
=& 4 \Bigg[4 \pi^2 \alpha^2
 + \frac{1}{\sqrt{2}} \pi \alpha G s \frac{M_Z^2 }{(s - M_Z^2)} [(g_v^f)^2 + 6 (g_A^f)^2]
\\
&
+ 8 G^2 s^2 \Big( \frac{M_Z^2}{s -M_Z^2} \Big)^2
[((g_v^e)^2 + (g_A^e)^2)^2 + 24 (g_v^e)^2 (g_A^e)^2] \Bigg] (1 + \cos^2 \theta)\\
&+ 24 \Bigg[
\sqrt{2} (g_A^f)^2 \pi  s \frac{M_Z^2 }{(s - M_Z^2)} \alpha G
+ (- 4)8 G^2 s^2 \Big( \frac{M_Z^2}{s -M_Z^2} \Big)^2 (g_v^e)^2 (g_A^e)^2
  \Bigg] \cos \theta\\
\end{align*}  
Now look at the coefficients separately and try to 'clean up'. The $1+\cos^2 \theta$-coefficient is 
\begin{align*}
\text{First } =& 4 \alpha^2 \Bigg[4 \pi^2
 + \frac{1}{\sqrt{2}} \pi  \frac{M_Z^2 }{(s - M_Z^2)} [g_v^2 + 6 g_A^2] \Big( \frac{s G}{\alpha} \Big)\\
& + 8   \Big( \frac{M_Z^2}{s -M_Z^2} \Big)^2
[(g_v^2 + g_A^2)^2 + 24 g_v^2 g_A^2] \Big( \frac{sG}{\alpha} \Big)^2 \Bigg] 
\end{align*}
The $\cos \theta$-coefficient is (we will now write $g_A = g_A^e$ and $g_v = g_v^e$)
\begin{align*}
\text{Second } =& 24 \alpha^2 \pi^2 \Bigg[  \frac{\sqrt{2} g_A^2}{\pi}  \frac{M_Z^2 }{(s - M_Z^2)} \Big( \frac{s G}{\alpha} \Big)
+ (- 4)8 \Big( \frac{sG}{\alpha} \Big)^2 \Big( \frac{M_Z^2}{s -M_Z^2} \Big)^2 g_v^2 g_A^2
  \Bigg]
\end{align*}
\end{flushleft}

\subsubsection{Differential cross section}
\begin{flushleft}
Again, the differential cross section is given by
\begin{align*}
\Big( \frac{d \sigma}{d \Omega} \Big)_{CM} = \frac{1}{2E_{\mathcal{A}}2 E_{\mathcal{B}} |v_{\mathcal{A}}-v_{\mathcal{B}}|} \frac{|\textbf{p}_1|}{(2 \pi)^2 4 E_{cm}}|\mathcal{M}(p_{\mathcal{A}}, p_{\mathcal{B}} \rightarrow p_1, p_2)|^2.
\end{align*}
\end{flushleft}

\subsection{Some identities}
\begin{flushleft}
The identities used to simplify the expressions in these calculations are as follows \cite{mandl2010quantum}
\begin{align}
\alpha &= \frac{e^2}{4 \pi}\\
M_Z &= (\frac{\alpha \pi}{G \sqrt{2}})^{1/2} \frac{2}{\sin 2 \theta_W}\\
g \sin \theta_W &= e 
\end{align}
\end{flushleft}




\bibliographystyle{plain}
\bibliography{project_2}

\end{document}